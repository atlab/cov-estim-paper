% Based on Template for PLoS
% Version 2.0 July 2014
%
%
% -- FIGURES AND TABLES
%
% DO NOT INCLUDE GRAPHICS IN YOUR MANUSCRIPT
% - Figures should be uploaded separately from your manuscript file. 
% - Figures generated using LaTeX should be extracted and removed from the PDF before submission. 
% - Figures containing multiple panels/subfigures must be combined into one image file before submission.
% See http://www.plosone.org/static/figureGuidelines for PLOS figure guidelines.
%
% Tables should be cell-based and may not contain:
% - tabs/spacing/line breaks within cells to alter layout
% - vertically-merged cells (no tabular environments within tabular environments, do not use \multirow)
% - colors, shading, or graphic objects
% See http://www.plosone.org/static/figureGuidelines#tables for table guidelines.
%
% For sideways tables, use the {rotating} package and use \begin{sidewaystable} instead of \begin{table} in the appropriate section. PLOS guidelines do not accomodate sideways figures.
%
% % % % % % % % % % % % % % % % % % % % % % % %
%
% -- EQUATIONS, MATH SYMBOLS, SUBSCRIPTS, AND SUPERSCRIPTS
%
% IMPORTANT
% Below are a few tips to help format your equations and other special characters according to our specifications. For more tips to help reduce the possibility of formatting errors during conversion, please see our LaTeX guidelines at http://www.plosone.org/static/latexGuidelines
%
% Please be sure to include all portions of an equation in the math environment, and for any superscripts or subscripts also include the base number/text. For example, use $mathrm{mm}^2$ instead of mm$^2$ (do not use \textsuperscript command).
%
% DO NOT USE the \rm command to render mathmode characters in roman font, instead use $\mathrm{}$
% For bolding characters in mathmode, please use $\mathbf{}$ 
%
% Please add line breaks to long equations when possible in order to fit our 2-column layout. 
%
% For inline equations, please do not include punctuation within the math environment unless this is part of the equation.
%
% For spaces within the math environment please use the \; or \: commands, even within \text{} (do not use smaller spacing as this does not convert well).
%
%
% % % % % % % % % % % % % % % % % % % % % % % %



\documentclass[10pt]{article}

% amsmath package, useful for mathematical formulas
\usepackage{amsmath}
% amssymb package, useful for mathematical symbols
\usepackage{amssymb}

% cite package, to clean up citations in the main text. Do not remove.
\usepackage{cite}

\usepackage{hyperref}

% line numbers
\usepackage{lineno}

% ligatures disabled
\usepackage{microtype}
\DisableLigatures[f]{encoding = *, family = * }

% rotating package for sideways tables
%\usepackage{rotating}

% If you wish to include algorithms, please use one of the packages below. Also, please see the algorithm section of our LaTeX guidelines (http://www.plosone.org/static/latexGuidelines) for important information about required formatting.
%\usepackage{algorithmic}
%\usepackage{algorithmicx}

% Use doublespacing - comment out for single spacing
%\usepackage{setspace} 
%\doublespacing


% Text layout
\topmargin 0.0cm
\oddsidemargin 0.5cm
\evensidemargin 0.5cm
\textwidth 16cm 
\textheight 21cm

% Bold the 'Figure #' in the caption and separate it with a period
% Captions will be left justified
\usepackage[labelfont=bf,labelsep=period,justification=raggedright]{caption}

% Use the PLoS provided BiBTeX style
\bibliographystyle{plos2009}

% Remove brackets from numbering in List of References
\makeatletter
\renewcommand{\@biblabel}[1]{\quad#1.}
\makeatother


% Leave date blank
\date{}

\pagestyle{myheadings}

%% Include all macros below. Please limit the use of macros.
\DeclareMathOperator{\Tr}{tr}
\newcommand{\loss}[1]{\mathcal L\left(#1\right)}
\newcommand{\T}{{\sf T}}
\newcommand{\E}[2][]{\mathbb E_{#1}\left[ #2\right]}    % expected value
\DeclareMathOperator*{\argmin}{arg\,min}
%% END MACROS SECTION


\begin{document}


% Title must be 150 characters or less
\begin{flushleft}
{\Large
\textbf{Improved estimation and interpretation of correlations in neural circuits}
}
% Insert Author names, affiliations and corresponding author email.
\\
Dimitri Yatsenko$^{1}$,
Kre\v{s}imir Josi\'{c}$^{2}$,
Alexander S.~Ecker$^{1,3,4}$,
Emmanouil Froudarakis$^{1}$,
R.~James Cotton$^{1}$,
Andreas S.~Tolias$^{1,5,\ast}$
\\
\bf{1} Department of Neuroscience, Baylor College of Medicine, Houston, TX, USA
\\
\bf{2} Department of Mathematics and Department of Biology and Biochemistry, University of Houston, Houston, TX, USA
\\
\bf{3}  Werner Reichardt Center for Integrative Neuroscience and Institute for Theoretical Physics, University of T\"ubingen, Germany
\\
\bf{4} Bernstein Center for Computational Neuroscience, T\"ubingen, Germany
\\
\bf{5} Department of Computational and Applied Mathematics, Rice University, Houston, TX, USA

$\ast$ E-mail: atolias@cns.bcm.edu
\end{flushleft}

% Please keep the abstract between 250 and 300 words
\section*{Abstract}
Ambitious projects aim to record the activity of ever larger and denser neuronal populations \emph{in vivo}.  Correlations in neural activity measured in such recordings can reveal important aspects of  neural circuit organization.  However, estimating and interpreting large correlation matrices is statistically challenging.  Estimation can be improved by regularization, \emph{i.e.}\;by imposing a structure on the estimate.  The amount of improvement depends on how closely the assumed structure represents dependencies in the data. Therefore, the selection of the most efficient correlation matrix estimator for a given neural circuit must be determined empirically.  Importantly, the identity and structure of the most efficient estimator informs about the types of dominant dependencies governing the system.
We sought statistically efficient estimators of neural correlation matrices in recordings from large, dense groups of cortical neurons.  Using fast 3D random-access laser scanning microscopy of calcium signals, we recorded the activity of nearly every neuron in volumes 200 $\mu$m wide and 100 $\mu$m deep (150--350 cells) in mouse visual cortex.  We hypothesized that in these densely sampled recordings, the correlation matrix should be best modeled as the combination of a sparse graph of pairwise partial correlations representing interactions between the observed neuronal pairs and a low-rank component representing common fluctuations and external inputs.  Indeed, in cross-validation tests, the covariance matrix estimator with this structure consistently outperformed other regularized estimators. The sparse component of the estimate defined a graph of interactions. These interactions reflected the physical distances and orientation tuning properties of cells: The density of positive `excitatory' interactions decreased rapidly with geometric distances and with differences in orientation preference whereas negative `inhibitory' interactions were less selective.  Because of its superior performance, this `sparse + latent' estimator likely provides a more physiologically relevant representation of the functional connectivity in densely sampled recordings than the sample correlation matrix.

% Please keep the Author Summary between 150 and 200 words
% Use first person. PLOS ONE authors please skip this step. 
% Author Summary not valid for PLOS ONE submissions.   
\section*{Author Summary}
It is now possible to record the spiking activity of hundreds of neurons at the same time.  A meaningful statistical description of the collective activity of these neural populations -- their `functional connectivity' -- is a forefront challenge in neuroscience.  We addressed this problem by identifying statistically efficient estimators of correlation matrices of the spiking activity of neural populations.  Various underlying processes may reflect differently on the structure of the correlation matrix:  Correlations due to common network fluctuations or external inputs are well estimated by low-rank representations, whereas correlations due to linear interactions between specific pairs of neurons are well approximated by their pairwise \emph{partial} correlations.  In our data obtained from fast 3D two-photon imaging of calcium signals of large and dense groups of neurons in mouse visual cortex, the best estimation performance was attained by decomposing the correlation matrix into a sparse network of partial correlations (`interactions') combined with a low-rank component. The inferred interactions were both positive (`excitatory') and negative (`inhibitory') and reflected the spatial organization and orientation preferences of the interacting cells.  We propose that  the most efficient among many estimators provides a more informative picture of the functional connectivity than previous analyses of neural correlations.


\section*{Introduction}
It is now possible to record the spiking activity of hundreds of neurons at the same time.  A meaningful statistical description of the collective activity of these neural populations -- their `functional connectivity' -- is a forefront challenge in neuroscience.  We addressed this problem by identifying statistically efficient estimators of correlation matrices of the spiking activity of neural populations.  Various underlying processes may reflect differently on the structure of the correlation matrix:  Correlations due to common network fluctuations or external inputs are well estimated by low-rank representations, whereas correlations due to linear interactions between specific pairs of neurons are well approximated by their pairwise \emph{partial} correlations.  In our data obtained from fast 3D two-photon imaging of calcium signals of large and dense groups of neurons in mouse visual cortex, the best estimation performance was attained by decomposing the correlation matrix into a sparse network of partial correlations (`interactions') combined with a low-rank component. The inferred interactions were both positive (`excitatory') and negative (`inhibitory') and reflected the spatial organization and orientation preferences of the interacting cells.  We propose that  the most efficient among many estimators provides a more informative picture of the functional connectivity than previous analyses of neural correlations.

\section*{Introduction}
\emph{Functional connectivity} is a statistical description of observed \emph{multineuronal} activity patterns not reducible to the response properties of the individual cells. Functional connectivity reflects local synaptic connections, shared inputs from other regions, and endogenous network activity. Although functional connectivity is a phenomenological description without a strict mechanistic interpretation, it can be used to generate hypotheses about the anatomical architecture of the neural circuit and to test hypotheses about the processing of information at the population level.

Pearson correlations between the spiking activity of pairs of neurons are among the most familiar measures of functional connectivity \cite{Averbeck:2006, Zohary:1994, Kohn:2005, Bair:2001, Ecker:2010}.  In particular, \emph{noise correlations}, \emph{i.e.}\;the correlations of trial-to-trial response variability between pairs of neurons, have a profound impact on stimulus coding \cite{Zohary:1994, Abbott:1999, Sompolinsky:2001, Nirenberg:2003, Averbeck:2006, Josic:2009, Berens:2011, Ecker:2011}. In addition, noise correlations and correlations in spontaneous activity have been hypothesized to reflect aspects of synaptic connectivity \cite{Gerstein:1964}.  Interest in neural correlations has been sustained by a series of discoveries of their nontrivial relationships to various aspects of circuit organization such as the physical distances between the neurons \cite{Smith:2008, Denman:2013}, their synaptic connectivity \cite{Ko:2011},  stimulus response similarity \cite{Bair:2001, Arieli:1995, Chiu:2002, Kenet:2003, Kohn:2005, Cohen:2008, Cohen:2009, Ecker:2010, Rothschild:2010, Ko:2011, Smith:2013b}, cell types \cite{Hofer:2011}, cortical layer specificity \cite{Hansen:2012, Smith:2013}, progressive changes in development and in learning \cite{Golshani:2009, Gu:2011, Ko:2013}, changes due to sensory stimulation and global brain states \cite{Greenberg:2008, Goard:2009, Kohn:2009, Rothschild:2010, Ecker:2010, Renart:2010}.

Neural correlations do not come with ready or unambiguous mechanistic interpretations. They can arise from monosynaptic or polysynaptic interactions, common or correlated inputs, oscillations, top-down modulation, and background network fluctuations, and other mechanisms \cite{Perkel:1967, Moore:1970, Shadlen:1998, Salinas:2001, Ostojic:2009, Rosenbaum:2011}. But multineuronal recordings do provide more information than an equivalent number of separately recorded pairs of cells. For example, the eigenvalue decomposition of the covariance matrix expresses shared correlated activity components across the population; common fluctuations of population activity may be accurately represented by only a few eigenvectors that affect all correlation coefficients. On the other hand, a correlation matrix can be specified using the \emph{partial correlations} between pairs of the recorded neurons. The partial correlation coefficient between two neurons reflects their linear association conditioned on the activity of all the other recorded cells \cite{Whittaker:1990}.  Under some assumptions, partial correlations measure conditional independence between variables and may more directly approximate causal effects between components of complex systems than correlations \cite{Whittaker:1990}. For this reason, partial correlations have been used to describe interactions between genes in functional genomics \cite{Schafer:2005, Peng:2009} and between brain regions in imaging studies \cite{Varoquaux:2012, Ryali:2012}. These opportunities have not yet been explored in neurophysiological studies where most analyses have only considered the distributions of pairwise correlations \cite{Zohary:1994, Bair:2001, Smith:2008, Ecker:2010}.

However, estimation of correlation matrices from large populations presents a number of numerical challenges. The amount of recorded data grows only linearly with population size whereas the number of estimated coefficients increases quadratically. This mismatch leads to an increase in spurious correlations, overestimation of common activity (\emph{i.e.}\;overestimation of the largest eigenvalues) \cite{Ledoit:2004}, and poorly conditioned partial correlations \cite{Schafer:2005}. The \emph{sample correlation matrix} is an unbiased estimate of the true correlations but its many free parameters make it sensitive to sampling noise. As a result, on average, the sample correlation matrix is farther from the true correlation matrix than some structured estimates.

Estimation can be improved through \emph{regularization},  the technique of deliberately imposing a structure on an estimate in order to reduce its estimation error \cite{Schafer:2005, Bickel:2006}. To `impose a structure' on an estimate means to bias (`shrink') it toward a reduced representation  with fewer free parameters, the \emph{target estimate}.   The optimal target estimate and the optimal amount of shrinkage can be obtained from the data sample either analytically \cite{Ledoit:2003, Ledoit:2004, Schafer:2005}  or by cross-validation \cite{Friedman:1989}. An estimator that produces estimates that are, on average, closer to the truth for a given sample size is said to be more \emph{efficient} than other estimators.

Although regularized covariance matrix estimation is commonplace in finance \cite{Ledoit:2003}, functional genomics \cite{Schafer:2005}, and brain imaging \cite{Ryali:2012}, surprisingly little work has been done to identify optimal regularization of neural correlation matrices.

Improved estimation of the correlation matrix is beneficial in itself. For example, improved estimates can be used to optimize  decoding of the population activity \cite{Friedman:1989}. But reduced estimation error is not the only benefit of regularization.  Finding the most efficient among many regularized estimators leads to insights about the system itself: the structure of the most efficient estimator is a parsimonious representation of the regularities in the data.

The advantages due to regularization increase with the size of the recorded population. With the advent of  big neural data \cite{Alivisatos:2013}, the search for optimal regularization schemes will become increasingly relevant in any model of population activity. Since optimal regularization schemes are specific to systems under investigation, the inference of functional connectivity in large-scale neural data will entail the search for optimal regularization schemes. Such schemes may involve combinations of heuristic rules and numerical techniques specially designed for given types of neural circuits.

What structures of correlation matrices best describe the multineuronal activity in specific circuits and in specific brain states?  More specifically, are correlations in the visual cortex during visual stimulation best explained by common fluctuations or by local interactions within the recorded microcircuit?

To address these questions, we evaluated four regularized covariance matrix estimators that imposed different structures on the estimate. The estimators are designated as follows:
\begin{description}
\item[$C_{\sf sample}$] -- sample covariance matrix, the unbiased, unregularized estimator.
\item[$C_{\sf diag}$] -- linear shrinkage of covariances toward zero, \emph{i.e.}\;toward a diagonal covariance matrix.
\item[$C_{\sf factor}$] -- a low-rank approximation of the sample covariance matrix, representing inputs from unobserved shared factors (latent units).
\item[$C_{\sf sparse}$] -- sparse partial correlations, \emph{i.e.}\;a large fraction of the \emph{partial} correlations between pairs of neurons are set to zero.
\item[$C_{\sf sparse+latent}$] -- sparse partial correlations between the recorded neurons \emph{and} linear interactions with a number of latent units.
\end{description}

First, we used simulated data to demonstrate that the selection of the optimal estimator indeed pointed to the true structure of the dependencies in the data.

We then performed a cross-validated evaluation to establish which of the four regularized estimators was most efficient for representing the population activity of dense groups of neurons in mouse primary visual cortex recorded with high-speed 3D random-access two-photon imaging of calcium signals. In our data, the sample correlation coefficients were largely positive and low.  We found that the best estimate of the correlation matrix was $C_{\sf sparse+latent}$.  This estimator revealed a sparse network of partial correlations (`{interactions'), between the observed neurons; it also inferred latent units exerting linear effects on the observed neurons. We analyzed these networks of partial correlations and found the following: Whereas significant noise correlations were predominantly positive, the inferred interactions had a large fraction of negative values possibly reflecting inhibitory circuitry.  Moreover, we found that these interactions exhibited a stronger relationship to the physical distances and to the differences in preferred orientations than noise correlations. In contrast, the inferred negative interactions were less selective.




% Results and Discussion can be combined.
% We only support three levels of headings, please do not create a heading level below \subsubsection.
\section*{Results}
\subsection*{Covariance estimation}
The covariance matrix is defined as
\begin{equation}\label{eq:true-covariance}
    \Sigma = \E{(x-\mu)(x-\mu)^\T},\quad \mu = \E{x}
    \end{equation}
    where $\E{\cdot}$ denotes expectation; the $p\times 1$ vector $x$ is a single observation of the firing rates of $p$ neurons in a time bin of some duration; and $\mu$ is the vector of expected firing rates. 

Given a set of observations $\{x(t): t\in T$\} of population activity, where $x(t)$ is a $p\times 1$ vector of firing rates in time bin $t$, and an independent unbiased estimate $\bar x$ of the mean activity, the \emph{sample covariance matrix},
\begin{equation}\label{eq:sample}
    C_{\sf sample} = \frac 1 n \sum\limits_{t\in T} (x(t)-\bar x)(x(t)-\bar x)^\T,
    \end{equation}
where $n$ is the number of time bins in $T$, yields an unbiased estimate of the covariance matrix so that $\E{C_{\sf sample}}=\Sigma$.

When the mean is estimated from the same sample, $C_{\sf sample}$ becomes biased toward zero.  However, in all the cases when the unbiasedness is required in our study, the mean will be estimated from a separate sample.

Given any covariance matrix estimate $C$, the corresponding correlation matrix $R$ is calculated by normalizing the rows and columns of $C$ by the square roots of its diagonal elements to produce unit entries on the diagonal:
\begin{equation}\label{eq:precision}
    R = \left(I\circ C\right)^{-\frac 1 2} C \left(I\circ C\right)^{-\frac 1 2},
\end{equation}
where $\circ$ denotes the entrywise matrix product (Hadamard product) and $I$ is the $p\times p$ identity matrix.

The \emph{partial correlation} between a pair of variables is the Pearson correlation coefficient of the residuals of the linear least-squares predictor of their activity based on all the other variables, excluding the pair \cite{Anderson:2003, Whittaker:1990}. Partial correlations figure prominently in probabilistic \emph{graphical modeling} wherein the joint distribution is explained by sets of two-way interactions \cite{Whittaker:1990}. For the multivariate Gaussian distribution, zero partial correlations indicate conditional independence of the pair, implying a lack of direct interaction \cite{Dempster:1972, Whittaker:1990}. More generally, partial correlations can serve as a measure of conditional independence under the assumption that most dependencies in the system include strong linear effects \cite{Whittaker:1990,Baba:2004}. As neural recordings become increasingly dense, partial correlations may prove useful as indicators of conditional independence (lack of functional connectivity) between pairs of neurons.

Pairwise partial correlations are closely related to the elements of the \emph{precision matrix}, \emph{i.e.}\;the inverse of the covariance matrix \cite{Dempster:1972,Whittaker:1990}. Zero elements in the precision matrix signify zero partial correlation between the two variables. Given the covariance estimate $C$, the matrix of partial correlations $P$ is computed by normalizing the rows and columns of the \emph{precision matrix} $C^{-1}$ to produce negative unit entries on the diagonal:
\begin{equation}\label{eq:partial}
    P = -\left(I\circ C^{-1}\right)^{-\frac 1 2} C^{-1} \left(I\circ C^{-1}\right)^{-\frac 1 2}
\end{equation}

As the size of recorded neuronal populations increases, without substantial increases in recording durations, the \emph{condition number} of the sample covariance matrix also increases, making the partial correlation estimates \emph{ill-conditioned}: small errors in covariance estimates translate into much greater errors in partial correlations after the inversion. With massively multineuronal recordings, partial correlations cannot be estimated without \emph{regularization} \cite{Ledoit:2004,Schafer:2005}.

We considered four regularized estimators based on distinct families of target estimates: $C_{\sf diag}$, $C_{\sf factor}$, $C_{\sf sparse}$, and $C_{\sf sparse+latent}$. In probabilistic models with exclusively linear dependencies, the target estimates of these estimators correspond to distinct families of graphical models (Fig.~\ref{fig:1} Row 1).

The target estimate of estimator $C_{\sf diag}$ is the diagonal matrix $D$ containing estimates of neurons' variances. Regularization is achieved by linear \emph{shrinkage} of the unbiased estimate $C_{\sf sample}$ toward $D$ controlled by the scalar \emph{shrinkage intensity} parameter $\lambda \in [0, 1]$:
\begin{equation}\label{eq:c-diag}
C_{\sf diag} = (1-\lambda) C_{\sf sample} + \lambda D
\end{equation}
The structure imposed by $C_{\sf diag}$ favors (performs better  on) populations   with no linear associations between the neurons (Fig.~\ref{fig:1} Row 1, A).  If sample correlations are largely spurious, $C_{\sf diag}$ is expected to be more efficient than other estimators.

Estimator $C_{\sf factor}$ approximates the covariance matrix by the factor model $L + D$, where $L$ is the $p\times p$ positive semidefinite matrix of a low rank and $D$ is a diagonal matrix of independent variances. This approximation is the basis for \emph{factor analysis} \cite{Anderson:2003}. The estimator,
\begin{equation}\label{eq:c-factor}
C_{\sf factor} = L + (1-\lambda)D + \lambda \bar D,
\end{equation}
has two hyperparameters: the rank of $L$ (\emph{i.e.}\;the number of latent factors) and the shrinkage intensity $\lambda$ to shrink the independent variances toward their mean value $\bar D$. The structure imposed by $C_{\sf factor}$ favors conditions in which the population activity is linearly driven by a number of latent factors that affect many cells while direct interactions between the recorded cells are insignificant (Fig.~\ref{fig:1} Row 1, B).

Estimator $C_{\sf sparse}$ is produced by approximating the sample covariance matrix by the inverse of a sparse matrix $S$:
\begin{equation}\label{eq:c-sparse}
C_{\sf sparse} = S^{-1}.
\end{equation}
Here $S$ is a sparse matrix, \emph{i.e.}\;one in which a large fraction of off-diagonal elements are set to zero.  The estimator has one hyperparameter that determines the sparsity (fraction of off-diagonal zeros) in $S$. The problem of finding the optimal set of non-zero elements of the precision matrix is known as \emph{covariance selection} \cite{Dempster:1972}. The structure imposed by $C_{\sf sparse}$ favors conditions in which neural correlations arise from direct linear effects (`interactions') between some pairs of neurons (Fig.~\ref{fig:1}} Row 1, C).

Estimator $C_{\sf sparse+latent}$ is obtained by approximating the sample covariance matrix by a matrix whose inverse is the sum of a sparse component and a low-rank component:
\begin{equation}\label{eq:c-sl}
C_{\sf sparse+latent} = (S - L)^{-1},
\end{equation}
where, as above, $S$ is a sparse matrix and $L$ is a low-rank matrix. The estimator has two hyperparameters: one to regulate the sparsity of $S$ and the other to regulate the rank of $L$. See Methods for a more detailed explanation. The structure imposed by $C_{\sf sparse+latent}$ favors conditions in which the activity of neurons is determined by linear effects between some observed pairs of neurons and linear effects from several latent units (Fig.~\ref{fig:1} Row 1, D) \cite{Chandrasekaran:2010,Ma:2013}.

We refer to the sparse partial correlations in estimators $C_{\sf sparse}$ and $C_{\sf sparse+latent}$ as `interactions'.

\subsection*{Simulation}
We next demonstrate how the most efficient among different regularized estimators can reveal the structure of correlations.
We constructed four families of $50\times 50$ covariance matrices, each with structure that matched one of the four regularized estimators (Fig.~\ref{fig:1} Row 2, A--D and Methods).  We used these covariance matrices as the ground truth in multivariate normal distributions with zero means and drew samples of various sizes. Even with relatively large sample sizes (\emph{e.g.}\;$n=500$), the sample correlation matrices were contaminated by sampling noise and their underlying structures were difficult to discern (Fig.~\ref{fig:1} Row 3).

The evaluation of any covariance matrix estimator, $C$, is performed with respect to a \emph{loss function} $\loss{C,\Sigma}$ to quantify its discrepancy from truth, $\Sigma$.  The loss function must attain its minimum when $C=\Sigma$.

In this study, we adopted the \emph{normal loss} function defined as
\begin{equation}\label{eq:loss}
    \loss{C,\Sigma} = \frac 1 p\left[\ln \det C + \Tr(C^{-1}\Sigma)\right]
\end{equation}

This loss function is related to the multivariate normal log-likelihood function $L(\Sigma|C_{\sf sample})$,
\begin{equation}
    \loss{X,Y} \equiv -\ln(2\pi) - \frac 2 {pn} L(X|Y)
\end{equation}
Normalization by $p$ and $n$ makes the values of the loss function comparable across different population sizes and sample sizes.

Although the two functions are similar in form, they differ conceptually: The log-likelihood $L(\Sigma|C_{\sf sample})$ is a function of the parameter $\Sigma$ given the sample covariance matrix $C_{\sf sample}$ whereas the loss function $\loss{C,\Sigma}$ expresses the deviation of an arbitrary estimate $C$ from the truth $\Sigma$. Minimum expected loss may be found at a point that is far removed from the point of maximum likelihood given the same data.

The choice of the normal loss function is motivated, in part, by mathematical convenience. We expect that the main conclusions of our study will not change qualitatively with other well behaved loss functions, such as Stein's entropy loss or quadratic loss \cite{James:1961, Fan:2008, Ledoit:2004, Schafer:2005}.

We drew 30 independent samples of sizes $n=250$, 500, 1000, 2000, and 4000 from each model and computed the \emph{excess loss} $\loss{C,\Sigma}-\loss{\Sigma,\Sigma}$ for each of the five estimators.  The hyperparameters of the regularized estimators were optimized by nested cross-validation using only the data in the sample.  All the regularized estimators produced better estimates (lower excess loss) than the sample covariance matrix.  However, estimators whose structure matched the true model outperformed the other estimators (Fig.~\ref{fig:1} Row 5).  Note that when the ground truth had zero correlations (Column A), $C_{\sf factor}$ performed equally well to $C_{\sf diag}$ because it correctly identified zero factors and only estimated the individual variances. Similarly, when the number of latent units was zero (Column C), $C_{\sf sparse+latent}$ performed equally well to $C_{\sf sparse}$ because it correctly identified zero latent units.

With increasing sample sizes, all estimators converged to the truth (zero excess loss).  However,
the discriminability between their performances increased with sample size (data not shown).

With empirical data, the ground truth, $\Sigma$, is not accessible and excess loss cannot be computed directly. Instead, the loss function may be estimated from the sample through \emph{validation}.  In a validation procedure, an independent \emph{testing sample} is set aside to compute an additional sample covariance estimate $C_{\sf sample}^\prime$ to validate the estimate $C$ computed from the  \emph{training sample}.   The estimate $C_{\sf sample}^\prime$ is used to compute the \emph{validation loss} $\loss{C,C_{\sf sample}^\prime}$, which approximates the loss  $\loss{C,\Sigma}$.

The normal loss (Eq.~\ref{eq:loss}) is particularly suitable because it is additive in its second argument, \emph{i.e.} 
\begin{equation*}\label{eq:additivity}
    \loss{C,Z_1} + \loss{C,Z_2} \equiv \loss{C,Z_1+Z_2}
\end{equation*}
Then, with fixed $C$, validation loss is an unbiased estimate of true loss:
\begin{equation*}
    \E{\loss{C,C_{\sf sample}^\prime}}=\loss{C,\E{C_{\sf sample}^\prime}}=\loss{C,\Sigma}.
\end{equation*}

The property of additivity does not hold for other popular loss functions such as Stein's entropy loss or various quadratic losses; their validation losses are biased estimates of the loss function and validation does not exactly evaluate a discrepancy from truth.

The loss function above assumes a single covariance matrix across all conditions in the dataset.  However, neuroscientists often estimate a common correlation matrix across multiple stimulus conditions when the variances of responses are conditioned on the stimulus \cite{Vogels:1989, Ponce:2013}. In this case, the stimulus-dependent means and variances are estimated from the training dataset along with a common correlation matrix. An adaptation of the normal loss function to this case is described in Methods.

Acquiring a separate testing sample is not practical. Instead, $K$-fold cross-validation can be used: The sample is divided into $K$ subsets of approximately equal size ($K=10$ in this study).  Then each subset is used as the testing sample with the other $K-1$ subsets serving as the training sample. The validation losses from all such `folds' are averaged to produce the \emph{cross-validation loss}.

Cross-validation loss accurately reproduced the relationships between the excess losses of the estimators (Fig.~\ref{fig:1} Row 6), confirming that cross-validation can evaluate true loss.

This simulation study demonstrated that, with sufficiently large sample sizes, imposing the correct type of structure on the estimate leads to greater estimation improvement than imposing other structures.

\subsection*{The $C_{\sf sparse+latent}$ estimator is most efficient in neural data}
We recorded the calcium activity of densely sampled populations of neurons in the supragranular layers in primary visual cortex of fentanyl-anesthetized mice using fast random-access 3D scanning two-photon microscopy during visual stimulation (Fig.~\ref{fig:2} A--B) \cite{Reddy:2005, Katona:2012, Cotton:2013}. This technique allowed fast sampling (100--150 Hz) from large numbers (150--350) of cells in a small volume of cortical tissue ($200\times200\times100$ $\mu$m$^3$) in layers 2/3 and 4 (Fig.~\ref{fig:2} C and D).  The firing rates were inferred using sparse nonnegative deconvolution \cite{Vogelstein:2010} (Fig.~\ref{fig:2} C). Only cells that produced detectable calcium activity were included in the analysis (see Methods).  First, 30 repetitions of full-field drifting gratings of 16 directions were presented in random order.  Each grating was displayed for 500 ms, without intervening blanks.  This stimulus was used to compute the orientation tuning of the recorded cells (Fig.~\ref{fig:2} D). To estimate the noise correlation matrix, we presented only 2 distinct directions in some experiments or 5 directions in others with 100--300 repetitions of each direction. Each grating lasted 1 second and was followed by a 1-second blank.  The traces were then binned into 150 ms intervals aligned on the stimulus onset for the estimation of the correlation matrix.   The sample correlation coefficients were largely positive and low (Fig.~\ref{fig:2} E and F). The average value of the correlation coefficient across sites ranged from 0.0065 to 0.051 with the mean across sites of 0.018 (Fig.~\ref{fig:5} D).

In these densely sampled populations, direct interactions between cells are likely to influence the patterns of population activity.  We therefore hypothesized that covariance matrix estimators that explicitly modeled the partial correlations between pairs of neurons ($C_{\sf sparse}$ and $C_{\sf sparse+latent}$) would have a performance advantage.  However, the observed neurons must also be strongly influenced by global activity fluctuations and by unobserved common inputs to the advantage of estimators that explicitly model common fluctuations of the entire population: $C_{\sf factor}$ and $C_{\sf sparse+latent}$.  If both types of effects are significant, then $C_{\sf sparse+latent}$ should outperform the other estimators.

To test this hypothesis, we computed the relative cross-validation loss of estimators  $C_{\sf sample}$, $C_{\sf diag}$, $C_{\sf factor}$, and $C_{\sf sparse}$ with respect to $C_{\sf sparse+latent}$ in $n=27$ imaged sites in 14 mice.  The hyperparameters of each estimator were optimized by nested cross-validation (See Fig.~\ref{supp:1} and  Methods). Indeed, the sparse+latent estimator outperformed the other estimators (Fig.~\ref{fig:3}). The respective median differences of the cross-validation loss were 0.039, 0.0016, 0.0029, and 0.0059 nats/cell/bin, significantly greater than zero ($p<0.01$ in each comparison, $n=27$ sites in 14 mice, Wilcoxon signed rank test).

The same evaluation based on the quadratic loss function,
\begin{equation}\label{eq:quad}
\loss{C,C_{\sf sample}^\prime}=\frac 1 {p^2}\Tr(C^{-1}C_{\sf sample}^\prime-I)^2,
\end{equation}
instead of the normal loss function reproduced the same relationship between the estimators (Fig.~S\ref{supp:02}). This suggests that the results of this study are robust to the choice of the loss function and do not depend on the assumption of gaussianity.

\subsection*{Structure of $C_{\sf sparse+latent}$ estimates}
We examined the composition of the $C_{\sf sparse+latent}$ estimates at each imaged site (Fig.~\ref{fig:4} and Fig.~\ref{fig:5}). Although the regularized estimates were similar to the sample correlation matrix (Fig.~\ref{fig:4} A and B), the corresponding partial correlation matrices differed substantially (Fig.~\ref{fig:4} C and D). The estimates separated two sources of correlations: a network of linear interactions expressed by the sparse component of the inverse and latent units expressed by the low-rank components of the inverse (Fig.~\ref{fig:4} E). The sparse partial correlations revealed a network that differed substantially from the network composed of the greatest coefficients in the sample correlation matrix (Fig.~\ref{fig:4} F, G, H, and I).

In the example site (Fig.~\ref{fig:4}), the sparse component had 92.8\% sparsity (or conversely, 7.2\% connectivity: $\mbox{connectivity}=1-\mbox{sparsity}$) with average node degree of 20.9 (Fig.~\ref{fig:4} G). The average node degree, \emph{i.e.}\;the average number of interactions linking each neuron, is related to connectivity as $\mbox{degree} = \mbox{connectivity}\cdot(p-1)$, where $p$ is the number of neurons. The low-rank component had rank 72, denoting 72 inferred latent units. The number of latent units increased with population size (Fig.~\ref{fig:5} A) but the connectivity was highly variable (Fig.~\ref{fig:5} B): Several sites, despite their large population sizes, were driven by latent units and had few pairwise interactions. This variability may be explained by differences in brain states and recording quality and warrants further investigation.

The average partial correlations calculated from these estimates according to Eq.~\ref{eq:partial} at all 27 sites were about 5 times lower than the average sample correlations (Fig.~\ref{fig:5} C). This suggests that correlations between neurons build up from multiple chains of smaller interactions. Furthermore, the average partial correlations were less variable: the coefficient of variation of the average sample correlations across sites was 0.45 whereas that of the average partial correlations was 0.29, with larger populations exhibiting more similar values of the average partial correlations than the smaller populations.

While the sample correlations were mostly positive, the sparse component of the partial correlations (`interactions') had a high fraction (28.7\% in the example site) of negative values (Fig.~\ref{fig:4} F). The fraction of negative interactions increased with the inferred connectivity (Fig.~\ref{fig:5} D), suggesting that negative interactions can be inferred only after a sufficient density of positive interactions has been uncovered.

Thresholded sample correlations have been used in several studies to infer pairwise interactions \cite{Golshani:2009, Feldt:2011, Malmersjo:2013, Sadovsky:2014}.  We therefore compared the interactions in the sparse component of $C_{\sf sparse+latent}$ to those obtained from the sample correlations thresholded to the same level of connectivity. The networks revealed by the two methods differed substantially. In the example site with 7.2\% connectivity in $C_{\sf sparse+latent}$, only 27.7\% of the connections coincided with the above-threshold sample correlations (Fig.~\ref{fig:4} F, H, and I). In particular, most of the inferred negative interactions corresponded to low sample correlations (Fig.~\ref{fig:4} F) where high correlations should be expected given the rest of the correlation matrix.

\subsubsection*{Relationship of $C_{\sf sparse+latent}$ to orientation tuning and physical distances}

We examined how the structure of the $C_{\sf sparse+latent}$ estimates related to the differences in orientation preference and to the physical distances separating pairs of cells (Fig.\;\ref{fig:6}).  Five sites with highest pairwise connectivities were included in the analysis. Partial correlations were computed using Eq.~\ref{eq:partial} based on the regularized estimate, including both the sparse and the latent component. Connectivity was computed as the fraction of pairs of cells connected by non-zero elements (interactions) in the sparse component of the estimate, distinguishing between the positive and negative connectivities.

First, we analyzed how correlations and connectivity depended on the difference in preferred orientations ($\Delta ori$) of pairs of significantly ($\alpha=0.05$) tuned cells. The partial correlations decayed more rapidly with $\Delta ori$ than did sample correlations ($p<10^{-9}$ in each of the five sites, two-sample $t$-test of the difference of the linear regression coefficients). Positive connectivity decreased with $\Delta ori$ ($p<0.005$ in each of the five sites, $t$-test on the logistic regression coefficient) whereas negative connectivity did not decrease (Fig.~\ref{fig:6} D): The slope in the logistic model of connectivity with respect to $\Delta ori$ was significantly higher for positive than for negative interactions ($p<0.04$ in each of the five sites, two-sample $t$-test of the difference of the logistic regression coefficient).

Second, we compared how correlations and connectivity depended on the physical distance separating pairs of cells. We distinguished between lateral distance, $\Delta x$, in the plane parallel to the pia, and vertical distance, $\Delta z$, orthogonal to the pia.  When considering the dependence on $\Delta x$, the analysis was limited to cell pairs located at the same depth with $\Delta z < 30\,\mu\mbox{m}$; conversely, when considering the dependence on $\Delta z$, only vertically aligned cell pairs with $\Delta x < 30\,\mu\mbox{m}$ were included. Again, the partial correlations decayed more rapidly both laterally and vertically than sample correlations ($p<10^{-6}$ in each of the five sites, for both lateral and vertical distances, two-sample $t$-test of the difference of the linear regression coefficients).
Positive connectivity decayed with distance ($p<10^{-6}$ in each of the five sites for positive interactions and $p<0.05$ for negative interactions, $t$-test on the logistic regression coefficient) (Fig.~\ref{fig:6} E), so that cells separated laterally by less than 25 $\mu\mbox{m}$ were 3.2 times more likely to be connected than cells separated laterally by more than 150 $\mu\mbox{m}$. Although the positive connectivity appeared to decay faster with vertical than with lateral distance, the differences in slopes of the respective logistic regression models were not significant with available data. The negative connectivity decayed slower with distance (Fig.~\ref{fig:6} E and F): The slope in the respective logistic models with respect to the lateral distance was significantly higher for positive than for negative connectivities ($p<0.05$ in each of the five sites, two-sample $t$-test of the difference of the logistic regression coefficients).

\section*{Discussion}

% You may title this section "Methods" or "Models". 
% "Models" is not a valid title for PLoS ONE authors. However, PLoS ONE
% authors may use "Analysis" 
\section*{Materials and Methods}



% Do NOT remove this, even if you are not including acknowledgments.

\section*{Acknowledgments}


%\section*{References}
% Either type in your references using
% \begin{thebibliography}{}
% \bibitem{}
% Text
% \end{thebibliography}
%
% OR
%
% Compile your BiBTeX database using our plos2009.bst
% style file and paste the contents of your .bbl file
% here.
% 
\bibliography{references.bib}

\section*{Figure Legends}
% This section is for figure legends only, do not include
% graphics in your manuscript file.

\begin{figure}
\caption{
{\bf Regularized estimators whose structure matches the true structure in the data are more efficient.}
     {\bf Row 1.} Graphical models of the target estimates of the four respective regularized covariance matrix estimators.  Recorded neurons are represented by the green spheres and latent units by the lightly shaded spheres.  Edges represent non-zero partial correlations, \emph{i.e.}\;`interactions'.
     {\bf Row 1, A}.  For estimator $C_{\sf diag}$, the target estimate is a diagonal matrix, which describes systems that lack linear dependencies.
     {\bf  Row 1, B.} For estimator $C_{\sf factor}$, the target estimate is a factor model (low-rank matrix plus a diagonal matrix), representing systems in which correlations arise due to common input from latent units.
     {\bf  Row 1, C}. For estimator $C_{\sf sparse}$, the covariance matrix is approximated as the inverse of a sparse matrix. This approximation describes systems in which correlations arise from a sparse set of  linear associations between the observed units.
     {\bf  Row 1, D}.  For estimator $C_{\sf sparse+latent}$, the covariance matrix is approximated as the inverse of the sum of a sparse matrix and a low-rank matrix. This approximation describes a model wherein correlations arise due to sparse associations between the recorded cells \emph{and} due to several latent units.
     \\
     {\bf Row 2:} Examples of $50\times 50$ correlation matrices corresponding to each structure: {\bf A.} the diagonal correlation matrix, {\bf B.} a factor model with four latent units, {\bf C.}  a correlation matrix with 67\%  off-diagonal zeros in its inverse, and {\bf  D.} a correlation matrix whose inverse is the sum of a rank-3 matrix (\emph{i.e.}\;three latent units) and a sparse matrix with 76\% off-diagonal zeros.
     \\
{\bf Row 3:} Sample correlation matrices calculated from samples of size $n=500$ drawn from simulated random processes with respective correlation matrices shown in Row 2.  The structure of the sample correlation matrix is difficult to discern by eye.
     \\
{\bf Row 4:} Estimates computed from the same data as in Row 3 using structured estimators of the correct type, optimized by cross-validation.  The regularized estimates are closer to the truth than the sample correlation matrices.
     \\
{\bf Row 5:} Excess losses (Eq.~\ref{eq:loss}) for the five estimators as a function of sample size. The error bars indicate the standard error of the mean based on 30 samples.  Estimators with structure that matches the true model converged to zero faster than the other estimators.
     \\
{\bf Row 6:} Validation losses for the five estimators relative to the matching estimator. Error bars indicate the standard error of the mean based on 30 samples.  Differences in validation loss approximate differences in true loss.
}
\label{fig:1}
\end{figure}


\begin{figure}
\caption{{\bf Acquisition of neural signals for the estimation of noise correlations.}
Visual stimuli comprising full-field drifting gratings interleaved with blank screens ({\bf A}) presented during two-photon recordings of somatic calcium signals using fast 3D random-access microscopy ({\bf B}).
{\bf C--F.} Calcium activity data from an example site.
{\bf C.} Representative calcium signals of seven cells, downsampled to 20 Hz, out of the 292 total recorded cells. Spiking activity inferred by nonnegative deconvolution is shown by red ticks below the trace.
{\bf D.} The spatial arrangement and orientation tuning of the 292 cells from the imaged site. The cells' colors indicate their orientation preferences. The gray cells were not significantly tuned.
{\bf E.} The sample noise correlation matrix of the activity of the neural population.
{\bf F.} Histogram of noise correlation coefficients in one site. The red line indicates the mean correlation coefficient of 0.020.
} \label{fig:2}
\end{figure}

\begin{figure}
\caption{
{\bf Performance of estimator $C_{\sf sparse+latent}$ with respect to the normal loss function (eq.~\ref{eq:loss}) relative to the other estimators: $C_{\sf sample}$, $C_{\sf diag}$, $C_{\sf factor}$, and $C_{\sf sparse}$.}
Covariance estimators $C_{\sf sample}$, $C_{\sf diag}$, $C_{\sf factor}$, and $C_{\sf sparse}$ produced consistently greater validation losses than $C_{\sf sparse+latent}$ ($p<0.01$ in each comparison, Wilcoxon signed rank test, $n=27$ sites in 14 mice). The box plots indicate the $25^{th}$, $50^{th}$, and $75^{th}$ percentiles with the whiskers extending to the minimum and maximum values after excluding the outliers marked with `+'.
} \label{fig:3}
\end{figure}

\begin{figure}
\caption{{\bf Structure revealed by $C_{\sf sparse+latent}$.}
{\bf A, B.} The regularized estimate $C_{\sf sparse+latent}$ closely approximates the sample correlation matrix $C_{\sf sample}$.
{\bf C, D.} The partial correlation matrices from the two estimates differ substantially.
{\bf E.} The partial correlation matrix of the regularized estimate is decomposed into a sparse component with 92.8\% off-diagonal zeros (bottom-left) and low-rank component of rank 72 (top-right).
{\bf F.} The sparse component of the regularized partial correlation matrix had little resemblance to the sample correlations: The gray region indicates the range of correlations containing 92.8\% of cells pairs, equal to the fraction of zeros in the sparse partial correlation matrix. Correlation coefficients outside this interval formed the network of greatest correlations.  This network differed from the sparse component of the $C_{\sf sparse+latent}$:  Only 27.7\% of the highest correlations coefficients outside the gray regions coincided with interactions inferred by $C_{\sf sparse+latent}$.
{\bf G.} A graphical depiction of the positive (green) and negative (magenta) sparse partial correlations as edges between observed neurons. The line density is proportional to the magnitude of the partial correlation.
{\bf H.} A subset of neurons from the center of the cluster shown in {\bf G} showing the sparse partial correlations.
{\bf I.} The same subset of neurons with edges indicating sample correlations thresholded to match the sparsity of the sparse partial correlation. These edges correspond to the sample correlation coefficients outside the gray region in panel F.
}
\label{fig:4}
\end{figure}
 
\begin{figure}
\caption{{\bf Properties of $C_{\sf sparse+latent}$ estimates from all imaged sites.}
Each point represents an imaged site with its color indicating the population size as shown in panels A and B. The example site from Figures \ref{fig:2} and \ref{fig:4} is circled in blue.
\\
{\bf A.} The number of inferred latent units \emph{vs.}~population size.
{\bf B.} The connectivity of the sparse component of partial correlations as a function of population size.
{\bf C.} The average sample correlations \emph{vs.}~the average partial correlations (Eq.~\ref{eq:partial}) of the $C_{\sf sparse+latent}$ estimate.
{\bf D.} The percentage of negative interactions vs.~connectivity in the $C_{\sf sparse+latent}$ estimates.
}
\label{fig:5}
\end{figure}


\begin{figure}
\caption{{\bf Dependence of sample correlations, regularized partial correlations, and connectivity inferred by $C_{\sf sparse+latent}$ on the differences in preferred orientations, $\Delta ori$, and physical distances, $\Delta x$ and $\Delta z$.} The error bars mark the standard errors of the means. Five sites with highest connectivity (see Fig.~\ref{fig:5} B) were selected for this analysis.
{\bf A--C.} Normalized mean sample correlations (black) and normalized mean partial correlations (red) estimated by $C_{\sf sparse+latent}$ across $n=5$ imaged sites. The values in each bin are normalized by the means across the entire site, shown in Fig.~\ref{fig:5} C, to make the effects more comparable across the sites.
{\bf A.} Mean partial correlations in $C_{\sf sparse+latent}$ depend on $\Delta ori$ more strongly than mean sample correlations.
{\bf B.} Mean partial correlations in $C_{\sf sparse+latent}$ depend on $\Delta x$ more strongly than mean sample correlations. Only horizontally aligned cell pairs with $\Delta z<30\,\mu m$ were considered.
{\bf C.} Mean partial correlations in $C_{\sf sparse+latent}$ depend on $\Delta z$ more strongly than mean sample correlations. Only vertically aligned cell pairs with $\Delta x<30\,\mu m$ were considered.    {\bf D--F.} Normalized positive connectivity (green) and normalized negative connectivity (dark red) inferred by the $C_{\sf sparse+latent}$ estimator in $n=5$ imaged sites.
Here ``connectivity'' refers to the fraction of the non-zero elements in the sparse component $S$ of the $C_{\sf sparse+latent}$ estimator, which describes inferred direct interactions between specific pairs of the recorded neurons. Positive and negative connectivities refer to the fractions of the positive and negative partial correlations computed from  $S$, respectively.  ``Normalized positive connectivity'' is the ratio of the positive connectivity for pairs meeting a given condition, \emph{e.g.}~similar tuning with $\Delta ori <15^{\circ}$, to the average connectivity over the entire site.  Normalized negative connectivity is computed similarly for the negative connectivity.  The average connectivity across sites is shown in Fig.~\ref{fig:5} B with only the five most connected sites included in the analysis.  The normalization made the effects of tuning and distance more comparable across sites.
{\bf D.} Positive connectivity decreases with $\Delta ori$ whereas negative connectivity does not.
{\bf E.} Positive connectivity decays with $\Delta x$ whereas negative connectivity does not, within the examined range.
{\bf F.} Positive connectivity decays with $\Delta z$. Negative connectivity does not decay for small values of $\Delta z$.
}
\label{fig:6}
\end{figure}

%\section*{Tables}
% 
% See introductory notes if you wish to include sideways tables.
%
% NOTE: Please look over our table guidelines at http://www.plosone.org/static/figureGuidelines#tables to make sure that your tables meet our requirements. Certain types of spacing, cell merging, and other formatting tricks may have unintended results and will be returned for revision.
%
%\begin{table}[!ht]
%\caption{
%\bf{Table title}}
%\begin{tabular}{|c|c|c|}
%table information
%\end{tabular}
%\begin{flushleft}Table caption
%\end{flushleft}
%\label{tab:label}
% \end{table}

\section*{Supporting Information Legends}
%
% Please enter your Supporting Information captions below in the following format:
%\item{\bf Figure SX. Enter mandatory title here.} Enter optional descriptive information here.
% 
%\begin{description}
%\item {\bf}
%\item {\bf}
%\end{description}

\end{document}

