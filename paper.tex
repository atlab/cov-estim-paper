\documentclass{frontiersSCNS} % for Science articles

\usepackage{url,lineno}
\usepackage{color}
\usepackage{enumerate}
\linenumbers
\usepackage{bbm}
\usepackage{centernot}

%%  Our commands
\DeclareMathOperator{\Tr}{tr}
\newcommand{\mcond}{\,\middle\vert\,}
\newcommand{\cond}{\,\vert\,}
\newcommand{\T}{\intercal}
\newcommand{\TODO}[1]{\emph{\tiny\color{Rhodamine}$\langle\langle$#1$\rangle\rangle$}}
\newcommand*\dif{\mathop{}\,d}

% Leave a blank line between paragraphs instead of using \\

\copyrightyear{}
\pubyear{}

\def\journal{Computational Neuroscience}
\def\DOI{}
\def\articleType{Research Article}
\def\keyFont{\fontsize{8}{11}\helveticabold }
\def\firstAuthorLast{Yatsenko {et~al.}} %use et al only if is more than 1 author
\def\Authors{
Dimitri Yatsenko\,$^{1,*}$, 
Alexander Ecker\,$^{2,1}$,
Emmanouil Foudarakis\,$^{1}$,
R.~James Cotton\,$^{1}$,
Kre\v{s}imir Josi\'{c}\,$^{3}$,
and Andreas S.~Tolias\,$^1$}

% Affiliations should be keyed to the author's name with superscript numbers and be listed as follows: Laboratory, Institute, Department, Organization, City, State abbreviation (USA, Canada, Australia), and Country (without detailed address information such as city zip codes or street names).
% If one of the authors has a change of address, list the new address below the correspondence details using a superscript symbol and use the same symbol to indicate the author in the author list.
\def\Address{
$^{1}$Department of Neuroscience, Baylor College of Medicine, Houston, TX, USA\\
$^{2}$Max Planck Institute of Biological Cybernetics, T\"ubingen, Germany\\
$^{3}$Department of Mathematics and Department of Biology and Biochemistry, University of Texas, Houston, TX, USA
}
% The Corresponding Author should be marked with an asterisk
% Provide the exact contact address (this time including street name and city zip code) and email of the corresponding author
\def\corrAuthor{Dimitri Yatsenko}
\def\corrAddress{Andreas Tolias Lab, Department of Neuroscience, Baylor College of Medicine, One Baylor Plaza, Houston, TX 77030, USA}
\def\corrEmail{yatsenko@cns.bcm.edu}

% \color{FrontiersColor} Is the color used in the Journal name, in the title, and the names of the sections

\begin{document}
\onecolumn
\firstpage{1}

\title[Improved estimates of neural correlations]{Improved estimates of neural correlations suggest detailed interactions in visual cortex}
\author[\firstAuthorLast ]{\Authors}
\address{}
\correspondance{}
\editor{}
\topic{Research Topic}

\maketitle
\begin{abstract}
Linear correlations between the spiking activity of pairs of neurons are among the most familiar and useful descriptive statistics of neural activity. Multineuronal recordings provide far richer information than the equivalent number of neuron pairs considered in isolation. For example, the full covariance matrix reveals the partial correlations between pairs of neurons, as well as correlated activity across the entire population. Covariance matrix estimates require large samples for convergence. Convergence  can be improved by regularization, \emph{i.e.}~by imposing some kind of structure.  The optimal regularization must be determined empirically since its performance depends on how the imposed structure captures the underlying interactions. For example, we can use low-rank parameterizations of the covariance matrix to account for correlated input onto many neurons. Conversely, if correlations are a result of a small number of  pairwise linear interactions between the observed neurons, we can sparsify the inverse covariance matrix. We can also use a `sparse+low rank’ inverse covariance representation to account for common fluctuations and pairwise interactions in the recorded population.  

To select the optimal structure of neural correlations in a local neural circuit, we compared the performance of several covariance estimators on the activity of 100--300 neurons in mouse visual cortex: sample covariance, Ledoit-Wolf covariance shrinkage, factor analysis, sparse inverse covariance, and sparse+low-rank inverse covariance. We inferred instantaneous firing rates in 200 ms bins from the somatic calcium signals acquired with fast 3D random-access two-photon microscopy.  Each covariance estimator was optimized and evaluated by cross-validation. As expected, covariance shrinkage reliably outperformed the sample covariance estimate. In turn, factor analysis-based estimates significantly outperformed covariance shrinkage. Yet sparse inverse covariance with or without an additional low-rank component significantly outperformed both factor analysis and shrinkage estimators. The superior performance of the sparse inverse covariance estimator suggests the relative importance of detailed network interactions over common diffuse input in the circuit we studied.


%As a primary goal, the abstract should render the general significance and conceptual advance of the work clearly accessible to a broad readership. References should not be cited in the abstract.
%See the Summary Table at \\ \url{http://www.frontiersin.org/}\texttt{\jour nal}\url{/authorguidelines} \\for abstract requirement and length according to article type.


\tiny
 \keyFont{ \section{Keywords:} population activity, neural circuits, functional connectivity, neural correlations, noise correlations, regularization, covariance estimation} %All article types: you may provide up to 8 keywords; at least 5 are mandatory.
\end{abstract}


\section{Introduction}
% For Original Research Articles, Clinical Trial Articles, and Technology Reports the introduction should be succinct, with no subheadings.
%
%For Clinical Case Studies the Introduction should include symptoms at presentation, physical exams and lab results.
%
Pearson correlations between the spiking activity of pairs of neurons, or simply \emph{neural correlations}, are among the most familiar descriptive statistics of neural population activity \cite{Averbeck:2006,Zohary:1994,Kohn:2005,Bair:2001,Renart:2010}.  For example, \emph{noise correlations}, \emph{i.e.}~the correlations of stimulus response variability between pairs of neurons, have been of particular interest.  Noise correlations can have profound theoretical implications for stimulus coding \cite{Zohary:1994,Abbott:1999,Averbeck:2006,Berens:2011,Ecker:2011}, and have been interpreted to indicate detailed and specific functional organization. Such interpretation is supported by a series of discoveries of nontrivial relationships between neural correlations and other aspects of circuit organization such as the physical distance separating the neurons \cite{Smith:2008,Denman:2013}, their synaptic connectivity and stimulus tuning similarity \cite{Kohn:2005,Ko:2011}, cortical layer specificity \cite{Hansen:2012,Smith:2013}, cell-type specificity (?), progressive changes in development and in learning \cite{Golshani:2009}, changes due to sensory stimulation and global brain states \cite{Goard:2009,Kohn:2009,Ecker:2010,Renart:2010}, and others.

However, neural correlations do not come with ready or unambiguous mechanistic interpretations.  Theoretical and simulation studies have shown that neural correlations may arise at various temporal scales from combinations of multiple underlying mechanisms.  These include direct synaptic interactions, common or correlated inputs, shared sensory noise, chains of multiple synaptic connections, oscillations, top-down modulation, and background network activity \cite{Perkel:1967b,Shadlen:1998,Salinas:2001,Ostojic:2009,Rosenbaum:2011}.

Early studies of neural correlations were based on measurements from isolated pairs of neurons and their impact on coding was extrapolated to entire populations \cite{Shadlen:1998,Zohary:1994}.  Multineuronal recordings allow estimation of covariance matrices of large populations of neurons.  Such estimates provide more information than the equivalent number of pairwise correlations assessed in isolation. Indeed, the correlation matrix is greater than the sum of its parts: it can be transformed into other representations that accentuate different aspects of the correlation structure and may suggest different mechanistic interpretations. For example, the eigenvalue decomposition of the covariance matrix expresses shared correlated activity components across the population; common fluctuations of population activity may be accurately represented by just a few principal components but will affect all correlation coefficients. In contrast, the off-diagonal elements of the inverse of the correlation matrix constitute scaled partial correlations between neuron pairs, which reflect their specific linear dependencies, after accounting for the activity of all the other recorded cells; a strong interaction between a pair of neurons may be expressed by a single partial correlation but its effects may propagate to multiple correlations and eigenvalues.   The inverse of the covariance matrix plays an important role in decoding schemes such as linear discriminant analysis, for example.  The mutliple  representations of the covariance matrix with their alternative interpretations add both complications and opportunities into the search for fundamental regularities in neural population activity. 

With large numbers of recorded cells, the usual estimations of the covariance matrix from finite recordings become ill-conditioned. Numerical instabilities arise because, as the amount of recorded data increases only linearly, the number of free coefficients to be estimated in any parameterization of the covariance matrix increases quadratically. 
For example, the large eigenvalues of large sample covariance matrices are biased upward while small eigenvalues are biased downward \cite{Ledoit:2004}. Similarly, the coefficients of the inverse covariance matrix require larger sample sizes to be estimated accurately from high-dimensional data than low-dimensional data.

With this study, we pursue two related goals: 
\begin{enumerate}
	\item Devise more efficient estimators of neural covariance matrices in recordings from large populations of neurons.
	\item Aid interpretation of neural covariance matrices.
\end{enumerate}

To select the best covariance matrix estimate for a specific neural circuit, we evaluate the performance of four estimators using different regularization schemes, each motivated by  a different hypothesis for the origin of neural correlations. \emph{Regularization} is the deliberate biasing the solution toward a simplified, low-dimensional (`sparse') \emph{target estimate} \cite{Schafer:2005,Bickel:2006}.
Regularized estimators allow reaching a favorable tradeoff between bias and variance in order to minimize the total error.
Strikingly, some improvement can be produced even when the target estimate is chosen arbitrarily, which is sometimes described as ``Stein's phenomenon'' after its discoverer \cite{Stein:1956}.   However, when a sparse target estimate does happen to match the dominant features of the true covariance matrix, it will induce relatively little bias while retaining low variance.  For example, some covariance estimators have been motivated by a specific model of the underlying process such as single-factor models of the stock market, for example \cite{Ledoit:2003}. 
Since the true covariance structure is not known in practice, the performance of estimators is evaluated by how well their performance generalized to new data, outside the training sample by cross validation, for example. 


We compared the performance of several covariance estimators for spatially compact groups of 150--300 neurons in layers 2--4 in mouse primary visual cortex during visual stimulation.   The performance of the estimators was evaluated by computing the mean squared error between the optimized covariance estimate fitted to training data and the non-regularized covariance matrix from a separate test data set.  Low-rank covariance estimates performed significantly better than shrinkage estimators, but estimators with sparse partial correlations were more efficient still. Typically, between 3 and 16\% of neuronal pairs were connected by non-zero partial interactions.  Mixed sparse estimators with sparse partial correlations and low-rank components performed comparably. 




%\begin{methods}
\section{Material \& Methods}
\subsection{Data acquisition}
Describe the two-photon experiment, refer to Fig.\;1

\subsection*{Basic concepts}
Let $x(t) \in \mathcal X, t=1\ldots,n$ denote a sample of consecutive observations of population activity of $p$ neurons in time bins $t$.  
The activity of each neuron is denoted by the real-valued firing rate, thus  $\mathcal X = \mathbb R^{p\times 1}$.  
For example, in our applicaiton, the firing rate is estimated from somatic calcium fluorescence signals with the mean stimulus response subtracted, making it possible for elements of $x(t)$ to take on negative and non-integer values. 

We do not assume that the observations $x(t)$ are independently distributed. Rather, the data generating process is assumed to be \emph{ergodic}, \emph{i.e.}\;described by a \emph{true distirbution} $F: \mathcal X \mapsto  \mathbb [0, 1]$ (cumulative) over long periods of time such that the \emph{empirical distribution} $\hat F_n$ from a given sample will convege to $F$ with increasing sample size $n$.

Formally, the emprical distribution is defined as \TODO{this may be unecessary, but I leave it for now.}
\begin{equation}
\hat F_n(x) = \frac 1 n \sum\limits_{t=1}^n \mathbf{1}(x \ge x(t))
\end{equation}
where $\mathbf 1(x \ge x(t))$ is the indicator function which equals 1 when all elements of $x$ are greater than the corresponding elements of $x(t)$ and 0 otherwise. \TODO{KJ: Are you using cumulative distributions to avoid binning? DY: Yes. Many theorems such as Glivenko-Catelli Theorem and Skorohod's Representation are proven using  cumulative representation. The empirical distribution is nearly always represented in its cumulative form. Expressing convergence is requires binning or using cumulative; the latter is simpler.}

And, for an ergodic process, $\max\limits_{x\in\mathcal X} \left|\hat F_n(x) - F(x)\right| \to 0$ as $n$ increases.

The true covariance matrix $\Sigma \in \Theta$ is defined as a function of $F$:
\begin{equation}
\Sigma = \int\limits_{z\in\mathcal X} (z - \mu)(z - \mu)^\T \dif F(z)
\quad\mbox{where}\;
\mu = \int\limits_{z\in\mathcal X} z \dif F(z)
\end{equation}
The domain $\Theta$ is the set of all positive-definite $p\times p$ matrices, which is a cone in $\mathbb R^{p\times(p+1)/2}$.

But the usual estimator of the true covariance matrix $\Sigma$ is  the sample covariance matrix:
\begin{equation}
\hat\Sigma_0 =  \frac 1 {n-c} \sum\limits_{t=1}^n (x(t) -\hat\mu) (x(t) - \hat\mu)^\T
\quad\mbox{where}\;
\hat\mu = \frac 1 n \sum\limits_{t=1}^n x(t)
\end{equation}
where $c$ is the sum of temporal correlations \TODO{there must be a better description of this}. For independent observations, $c=1$;  $c>1$ when nearby samples are correlated; the value of $c$ can be estimated from the data.
\TODO{Under the Gaussian loss, $\mathcal L_g$ (\ref{eq:GaussLoss}) the scaling is irrelevant, so we will not spend any time talking about the scaling.  But it would be important if we used $\mathcal L_e$ (\ref{eq:MSE}).} 

$\hat\Sigma_0$ is designed to be unbiased such that $\mathbb E\left[\hat\Sigma_0\right]=\Sigma$.
Here and througout, $\mathbb E[\cdot]$ denotes the expected value under the  unknown true distribution $F$. All variables with a hat (\emph{e.g.}\;$\hat \Sigma_0,\hat \mu$) are functions of the empirical distribution $\hat F_n$, which itself is a random variable.

%Nonlinear functions of an unbiased estimate can be biased: $\mathbb E\left[\hat\Sigma_0\right]=\Sigma \centernot\implies  E\left[\varphi(\hat\Sigma_0)\right]=\varphi(\Sigma)$.  For example, if $u_{\max}(S)$ is the largest eigevalue of square matrix $S$, then $\mathbb E\left[u_{\max}(\hat\Sigma_0)\right] > u_{\max}(\Sigma)$. Some covariance matrix estimators are designed to correct for the eigenspectrum bias at the cost  of adding bias to covariance coefficients \cite{Ledoit:2004}.




\subsection*{Loss and risk}
The optimization of a covariance matrix estimate $\hat\Sigma$ is performed with respect to a \emph{loss function} $\mathcal L(\hat\Sigma,\Sigma)$, which expresses the discrepancy between $\hat\Sigma$ and $\Sigma$ and attains its minimum when $\hat\Sigma=\Sigma$.  
Then \emph{excess loss}  
\begin{equation}
\ell(\hat\Sigma,\Sigma) = \mathcal L(\hat\Sigma,\Sigma)-\mathcal L(\Sigma,\Sigma)
\end{equation}
assumes zero at its minimum.

A particularly useful loss function is the mean squared error (MSE), which is proportional to the square of the Frobenius  norm $\|\cdot\|_F$ of the difference between the two matrices: 
\begin{equation}\label{eq:MSE}
\mathcal L_e(\hat\Sigma,\Sigma) =\frac 1 p \|\hat\Sigma-\Sigma\|_F^2 = \frac 1 p \Tr\left((\hat \Sigma-\Sigma)(\hat\Sigma-\Sigma)^\T\right)
\end{equation}
Since $\mathcal L_e(\Sigma,\Sigma) \equiv 0$, the MSE is its own excess loss: $\ell_e(\hat\Sigma,\Sigma) \equiv \mathcal L_e(\hat\Sigma,\Sigma)$.

The Gaussian loss function $\mathcal L_g$  arises from the theory of multivariate normal distributions. When observations are identically and independently distributed according to a multivariate normal distribution with zero means, the log likelihood of the covariance matrix $\Sigma$ with $\hat\Sigma = \frac 1 n \sum\limits_{t=1}^n x(t) x(t)^\T$ is  
\begin{equation}
L\left(\Sigma \mid \hat\Sigma\right) = -\frac n 2 \ln(2\pi) - \frac n 2 \ln \det \Sigma - \frac n 2 \Tr(\Sigma^{-1} \hat \Sigma)
\end{equation}
Then $\mathcal L_g$ is constructed by rescaling $L\left(\Sigma \mid \hat\Sigma\right)$ and dropping the constant term:
\begin{equation}\label{eq:GaussLoss}
\mathcal L_g(\hat\Sigma,\Sigma) 
=  -\frac 2 {pn} L\left(\Sigma \mid \hat\Sigma \right) - \frac 1 p \ln(2\pi) 
\equiv  \frac 1 p\left(\ln \det \hat \Sigma + \Tr(\hat \Sigma^{-1}) \right) 
\end{equation}
The corresponding excess loss 
\begin{equation}
\ell_g(\hat\Sigma,\Sigma) = \mathcal L_g(\hat\Sigma,\Sigma) - \mathcal L_g(\Sigma,\Sigma)  
= \frac 1 p \left(-\ln \det (\hat \Sigma^{-1} \Sigma) + \Tr(\hat \Sigma^{-1}\Sigma)\right) - 1
\end{equation}
is known as \emph{entropy loss} \cite{James:1961}. \TODO{The order of the operands is reversed. Check that it's okay.}

Despite the fact that entropy loss is derived from normal theory, the choice of a loss function is not equivalent to assuming a specific form of $F$. The loss function measures the discrepancy between distribution parameters rather than the distance between the distributions themselves. \TODO{explain Bregman divergence?} 

The expected value of excess loss is the \emph{estimator risk}:
\begin{equation}\label{eq:risk}
r = \mathbb E\left[\ell(\hat\Sigma,\Sigma)\right]
\end{equation}
The estimator risk is the primary quality criterion for covariance estimation. Estimator $\hat\Sigma_a$ is considered more \emph{efficient} than estimator $\hat\Sigma_b$ if $\hat\Sigma_a$ has lower risk for the given sample than $\hat\Sigma_b$.   \TODO{explain that the risk depends not only on $\hat\Sigma$ but also on the distribution of $\Sigma$ in the specific domain.}

\subsection*{Cross validation}
In practice, the true covariance matrix $\Sigma$ is not accessible and the estimator risk $r$ must be estimated from data. This requires a separate \emph{validation} empirical distribution $\hat F_m^\prime$ of $m$ observations sampled from $F$ independently of the \emph{training} distribution $\hat F_n$. 
Then let  $\hat \Sigma_0^\prime$ be the sample covariance obtained from $\hat F_m^\prime$. 

Then the \emph{empirical loss} of $\hat\Sigma$ is $\mathcal L(\hat\Sigma,\hat\Sigma_0)$ and its expectation is the \emph{empirical risk}  
\begin{equation}
\hat r = \mathbb E\left[\mathcal L(\hat\Sigma,\hat\Sigma_0^\prime)\right]
\end{equation}

 The two loss functions $\mathcal L_e(\hat\Sigma,\Sigma)$ and $\mathcal L_g(\hat\Sigma,\Sigma)$ are particularly suitable for risk estimation thanks to their linearity with respect to $\Sigma$ in the sense that 
\begin{equation}
\mathcal L\left(\hat\Sigma,\alpha S_1 + (1-\alpha)S_2\right) 
\equiv 
\alpha\mathcal L(\hat \Sigma,S_1) + (1-\alpha)\mathcal L(\hat \Sigma,S_2)
\end{equation}
which allows bringing the expectation inside the empirical loss function:
\begin{equation}
\hat r = 
\mathbb E\left[ \mathcal L(\hat\Sigma, \hat\Sigma_0^\prime) \right] 
=
\mathbb E\left[ \mathcal L\left(\hat\Sigma, \mathbb E\left[\hat\Sigma_0^\prime\right]\right) \right] 
=
\mathbb E\left[ \mathcal L(\hat\Sigma, \Sigma) \right] 
= 
r + \mathcal L(\Sigma,\Sigma)
\end{equation}
This means, that the empirical loss $\mathcal L(\hat\Sigma,\hat\Sigma_0^\prime)$ is an unbiased estimate of the estimator risk $r$ (up to a constant offset): minimization of the empirical loss implies minimization of the true estimator risk. 

Recording a separate independent validation sample is usually not sensible.  Instead, the recorded data are split into a training sample and validation sample. In $K$-fold \emph{cross-validation} the data are split into $K$ roughly equal-sized subsets. Each subset then successively serve as the testing sample while the remainder of the data serves as the training sample.  This procedure produces $K$ estimates of the  risk, which are then averaged together to produce a better averaged estimate.


\subsection*{Bias/variance decomposition}
The estimator risk (\ref{eq:risk}) can be decomposed into \emph{approximation error} or \emph{bias}   
\begin{equation}
b^2 = \ell \left( \mathbb E[\hat \Sigma],\Sigma\right)
\end{equation}
and \emph{estimation error} or \emph{variance}
\begin{equation}
\varepsilon^2 = \mathbb E \left[ \ell\left(\hat \Sigma, 
\mathbb E[\hat \Sigma]\right) \right]
\end{equation}

Under the MSE loss function $\mathcal L_e$ (\ref{eq:MSE}), the decomposition is the simple sum $r = b^2 + \varepsilon^2$. 
Under other loss functions, $r$ is an increasing function of both $b^2$ and $\varepsilon^2$ although not generally a simple sum.  
\TODO{As far as I know, the terms `bias' and `variance' are only applicable under a quadratic loss function  like $\mathcal L_e$.  Under other loss functions, we may need to use approximation/estimation error.} 

The sample covariance matrix $\hat\Sigma_0$ has zero bias but high variance. Other estimators may be biased but have lower estimation error and potentially lower risk.


\subsection*{Regularization}

\emph{Regularization} is the deliberate biasing (`shrinkage') of the unbiased estimate toward a less variable low-dimensional \emph{target estimate} $\hat T$ in order to minimize the estimator risk by striking a favorable balance between bias and variance.
A regularized covariance matrix estimator must solve two problems: (a) choose and fit the target covariate estimate $\hat T$ and (b) mix $\hat\Sigma_0$ with $\hat T$ by the optimal amount. 
Regularized covariance matrix estimates can be generally expressed as 
\begin{equation}
\hat\Sigma_{d,\lambda} = mix(\Sigma_0,\hat T_{\hat d},\hat\lambda) 
\end{equation}
where $\hat d$ indicates the choice of the target estimate from the family of target estimates $\hat T_d$. $mix(\cdot,\cdot,\hat\lambda)$ is the mixing function of $\hat\Sigma_0$ and $\hat T_{\hat d}$ in proportion controlled by $\hat \lambda$. Both $\hat d$ and $\hat \lambda$ must be calculated from the training sample. 

A number of regularization schemes have been developed, which differ by their target estimates and mixing functions.    Curiously, \emph{some} improvement can be produced with an arbitrary target estimate as long as its variance is lower than that of $\hat\Sigma_0$.  The perplexing phenomenon that estimates can be improved by a bias toward an arbitrary less variable target is known as \emph{Stein's paradox} \cite{Efron:1977}.   \TODO{This point can be untuitive.  For example, \cite{Varoquaux:2012} misrepresents the effect of regularization by stating that regularization removes the upward bias of correlations in the sample covariance matrix. In reality, regularization should start with an unbiased estimator and bias it toward a low-variance target so that the estimation risk is reduced.} However, if a family of target estimates can capture the underlying regularities in the covariance structure of a specific system with a small number of parameters, biasing the estimate toward these targets ought to produce a greater reduction in estimator risk. 

In this study, we evaluate four regularized estimators whose target estimates correspond to the graphical models in Figure \ref{fig:02}: diagonal (A), multifactor (B), sparse partial correlations (C), and sparse  partial correlations with latent units (D).  The following describe each regularization scheme separately.

\begin{figure}[htp]
\centering
\includegraphics[width=0.5\textwidth]{figures/Figure2.pdf}
\caption{
Graphical models corresponding to the low-dimensional targets of the four regularization schemes used in the paper.
\textbf{A}: A diagonal matrix corresponds to a Gaussian graphical model with no dependencies. 
\textbf{B}: In factor analysis, observed nodes are assumed to be influenced by several latent units (``factors") but are otherwise independent. 
\textbf{C}: In the Gaussian graphical model (also known as the Gaussian Markov Field), correlations arise from sparse pairwise linear interactions between visible units. 
\textbf{D}: In the Gaussian graphical model with latent units, correlations arise  between pairs of nodes 
}\label{fig:02}
\end{figure}


\subsection{Simulation}
To illustrate,

\textbf{Figure 2.}{
Simulation results. 
\textbf{A}: A $100 \times 100$ covariance matrix without a low-dimensional structure, a sample covariance with n=2000, distributions of true and sample correlation coefficients, eigenspectra of true and sample covariances.
\textbf{B}: A similar covariance matrix with 8 factors.
\textbf{C}: A similar covariance matrix with sparse inverse. 
\textbf{D}: A similar covariance matrix with sparse + low-rank inverse with 3 latent units.
\textbf{E}: Risk convergence rates
}\label{fig:02}


\section{Results}

\subsection*{Covariance estimation}

Our goal is to estimate the covariance matrix, $\Sigma$, of a population of $n$ neurons. For small populations or in the limit of infinite data this can be done in a straightforward manner by using the sample covariance estimator
\begin{equation}
\hat \Sigma_0 = \frac 1 \nu \sum\limits_{t=1}^n (x(t)-\mu)(x(t)-\mu)^\T, 
\end{equation}
where $x(t)$ are sequential observations of population activity and $\nu$ is the number of degrees of freedom ($\nu=n-1$ if observations are independent).

However, for large population sizes and limited data this approach becomes ineffective at capturing the true structure of the covariance matrix (REFS). The number of parameters to be estimated grows quadratically with the population size. Thus, to obtain a "good" estimate of the true covariance, we need to rely on regularization -- the deliberate biasing of the estimate toward a low-dimensional, less variable target estimate \cite{Bickel:2006,Ledoit:2004}. Various such targets exist, each expressing a different idea about what a parsimonious model would look like. 

We considered four different estimators, based on a few simple concepts for regularizing the covariance: independence, low-rank structure and sparse interactions. We optimized all four estimators with respect to Gaussian log-likelihood (see Discussion for a more in-depth justification of this loss function) and used cross-validation to determine hyper-parameters.

The first and probably simplest regularized estimator is shrinkage towards a diagonal matrix
\begin{equation}
\hat \Sigma_{\rm diag} = (1-\lambda) \hat \Sigma_0 + \lambda D,
\end{equation}
where $D$ is a diagonal matrix and the mixing proportion, $\lambda$, is determined by cross-validation. This estimator expresses the idea that in the absence of evidence for strong correlations between cells we assume they are independent (Fig.~1A). This approach is sometimes also referred to as ridge regression. 

The second estimator we consider is shrinkage towards a low-rank matrix
\begin{equation}
\hat \Sigma_{\rm low-rank} = (1-\lambda) \hat \Sigma_0 + \lambda T,
\end{equation}
where $T = LL^\T + \Psi$ is a factor analysis model (REFS) consisting of a low-rank component $L$ with rank $d < n$ and a diagonal matrix $\Psi$. This estimator expresses the idea that the fluctuations in population activity are driven by a small number of unobserved sources that affect many cells (Fig.~1B). 

The third estimator is based on sparsifying the inverse covariance (precision) matrix. It expresses the idea that, similar to synaptic connections, interactions between neurons are sparse (Fig.~1C). The estimator produces a low-dimensional approximation to the covariance that has many zeros in its inverse. This approximation problem is also known as covariance selection. The covariance matrix estimate is given by
\begin{equation}
\hat\Sigma = \hat S^{-1}
\quad\mbox{with}\quad
\hat S = \argmin\limits_{\|S\|_0 \le \rho} \loss{S^{-1},\hat\Sigma_0}   
\end{equation}
where $\mathcal L$ is a loss function (see below) and $\|S\|_0\le\rho$ signifies the constraint that $S$ has at most $\rho$ non-zero coefficients. While solving the problem in this form is computationally challenging, relaxing the $L_0$ norm to the $L_1$ norm converts it into one of convex optimization \cite{Donoho:2000}, which is computationally much more tractable. The resulting algorithm, which is known as graphical lasso \cite{Meinshausen:2006,Yuan:2007,Banerjee:2008,Friedman:2008}, produces the following estimator  
\begin{equation}
\hat\Sigma_{\rm sparse}^\mathcal{C} = \hat S^{-1}
\quad\mbox{with}\quad
\hat S = \argmin \loss{S^{-1},\hat\Sigma_0} + \lambda \|S\|_1,
\end{equation}
where $\|S\|_1$ is the $L_1$ norm of the precision matrix (see Methods for details) and $\lambda$ is again determined by cross-validation.

Finally, we consider a fourth estimator, which combines common inputs with sparse interactions (Fig.~1D). For this estimator the covariance is
\begin{equation}
\hat\Sigma_{\rm combined} = (S + LL^\T)^{-1},
\end{equation}
where, as above, $S$ is sparsified using $L_1$ norm and $L$ is a low-rank matrix with the rank determined by cross-validation.



\subsection*{Simulation using toy data}

To verify our approach and to illustrate the performance of our four regularized estimators and the sample covariance estimator, we constructed five model populations with different underlying structures. Each population contained 100~neurons. The first four populations matched the low-dimensional structure of our four estimators: independent (Fig.~2\,A), low-rank (Fig.~2\,B), sparse inverse (Fig.~2\,C) and low-rank combined with sparse inverse (Fig.~2\,D). \Acomment{Integrate schematics from Fig.~1 into this figure (make it the first row) for better correspondence} In addition, we considered a fifth population that had no low-dimensional structure (Fig.~2\,E).

To evaluate the performance of the different estimators, we computed the excess loss for all combinations of model populations and estimators, including the sample covariance. The first striking observation is that in all cases all four regularized estimators performed substantially better than the sample covariance (Fig.~2, fourth row). In particular, this is even true for the case where the population did not have any low-dimensional structure at all (Fig.~2\,E). While it may appear surprising at first sight, this counter-intuitive phenomenon is known as \emph{Stein's phenomenon} or \emph{Stein's paradox} \cite{Efron:1977}, named after its discoverer Charles Stein \cite{Stein:1956}. A common misconception about regularization is that its effect depends on accurate prior knowledge about the structure of the data. However, substantial improvement can be attained by shrinking the unbiased estimate toward an arbitrary target as long as the target is less variable than the unbiased estimator. The more accurate description of regularization is as of the optimal tradeoff between estimation and approximation error -- the so-called ``bias-variance tradeoff''.

Thus, if taken in isolation a regularized estimator improves the estimate we should not interpret this result to suggest that the estimator's target has the same low-dimensional structure as the data-generating process. However, when comparing multiple estimators against each other, the one whose target estimate most closely matches the true value with the smallest number of parameters will reduce the estimation error with the least increase in approximation error. Indeed, in all four toy examples with a low-dimensional structure, the estimator with the matching regularization target performed best (Fig.~2\,A--D, fourth row). Since the estimator combining low-rank and sparse interactions combines to low-dimensional targets and includes the simpler estimators as special cases, it performed almost as well even when the low-dimensional structure did not include either a low-rank component or sparse interactions (Fig.~2\,B,\,C). \Acomment{May want to include a smaller sample size as well since here the more reduced estimators (B, C) may actually outperform estimator D when the population structure matches their target and there is little data (say, 1.5x number of neurons)} 

Since the above evaluations of excess loss require knowledge of ground truth, this analysis can be done only on toy data. However, an unbiased empirical estimate of the excess loss exists (see Methods, Eq.~\ref{eq:validationLoss} \Acomment{This may need a brief sentence about the intuition what's the difference, or a reference to the relevant methods section}).  To verify that our approach should also work in a realistic situation without access to ground truth, we also computed the empirical loss. Indeed, without access to ground truth, this analysis revealed a pattern of results that reproduced the results obtained with access to ground truth above (Fig.~2, last row). Because validation loss is computed by comparing estimates to noisy sample covariance matrices from smaller validation sets, empirical loss is a much noisier measurement than excess loss. In addition, it does not converge to zero with increasing sample size as excess loss does.




\subsection*{Covariance estimation on neural data}


We recorded the calcium activity of dense populations of neurons in the supragranular layers in primary visual cortex of anesthetized mice using fast random-access 3D scanning two-photon microscopy \cite{Stosiek:2003,Reddy:2005}. We presented numerous repetitions of full-field drifting gratings (Fig. 1A and 1B) to the eye contralateral to the imaged site. This technique allowed us to record from a large number (150--350) of cells in a small volume of cortical tissue ($200\times200\times100$ $\mu$m$^3$) in layers 2/3 and 4. We deconvolved somatic calcium signals using sparse nonnegative deconvolution \cite{Vogelstein:2010} (Fig.\;1C and 1D) and subtracted the average stimulus response to remove effects of the stimulus. From this residual response we computed the sample noise covariance matrix (Fig.\;1E).

In such highly localized populations both direct interactions between cells and common diffuse inputs are likely to contribute to the overall population variability. At the same time, most correlations are relatively small (Fig.~1\,E), suggesting that a simple shrinkage towards independence may provide a sufficiently well regularized estimate. By comparing the performance of our differently regularized estimators we can gain insights into which of these aspects are important in our data.

As expected, all regularized estimators outperformed the sample covariance estimator substantially (Fig.~4) \Acomment{I think we should show that}. Say something about how shrinkage, low-rank and sparse inverse relate \Acomment{Why do we do pairwise comparisons instead of just showing median (across sites) log loss relative to the combined estimator? I've seen many people do this and it provides an ordering. I feel like we should come up with a way of ordering them somehow, even if it's under certain assumptions that may not be entirely correct. I'm pretty sure if we don't do it the reviewers will bring it up anyway...}. Finally, the combined sparse and low-rank estimator dominated all others significantly (Fig.~4), showing that both hidden units and direct interactions are important in our data. The improvement from the low-rank estimator to the combined one is much larger than that from the sparse inverse to the combined one, suggesting that in this dataset direct interactions contribute more strongly to the correlation structure than hidden units do.



\subsection*{Relationship between functional covariance structure and circuit architecture}


\begin{itemize}
\item Linear and partial correlations versus spatial separation (lateral and vertical). Discuss whether to include or remove thresholded correlations.
\item Magnitude of common input versus spatial location in the volume (lateral and vertical separation from center).
\item Linear and partial correlations versus orientation preference
\item Distribution of sparsity over sites
\item Distribution of number of hidden units over sites
\item Distribution of eigenvalues for sample covariance versus combined estimator
\item etc. etc. 

\end{itemize}







\section{Discussion}
\hl{The discussion section is not ready for any feedback. Currently it's just a trashbin with clips from scrapped versions of the introduction. -- Dimitri}

Fundamental questions in systems neuroscience concern the relationship between the function of neural microcircuits and their cytoarchitecture: circuit topology, cell types, and patterns of synaptic connectivity. 

Considerable progress has been made in sensory areas where the functional characterizations of cells are defined by their reproducible responses to external stimuli (reviewed in \citep{Reid:2012}). 

Many key questions in neuroscience revolve around the relationship between the cytoarchitecture of neural microcircuits and the functional organization of their activity \citep{Reid:2012}.  Investigators in this line of inquiry aim to construct networks of functional associations within groups of neurons inferred from observations of their activity under a variety of conditions.  The inferred functional network  could then related to the synaptic connectivity patterns, cell types, cell properties, or the cells' spatial arrangment in the attempt to uncover organizational principles of neural computation. In addition, the network itself can serve as subtrate for subsequent graph-theoretical analysis \citep{Feldt:2011}

Advances in electrophysiology and optical imaging have enabled simultaneous recordings from dozens to hundreds or thousands of cells. 
In particular, recent advances in two-photon imaging of calcium signals have allowed in vivo recordings from nearly every neuron brain-wide in zebrafish \citep{Leung:2013,Ahrens:2013} and in a 3D volume of a few hundred microns in diameter in mouse neocortex \citep{Katona:2012,Cotton:2013}.    
Ambitious projects currently under way aim to record the spiking activity of large fractions of cells from entire circuits and systems in behaving animals \citep{Alivisatos:2012}.  Direct observations of the population activity  of entire circuits open new possibilities for incisive statistical descriptions of the functional connectivity in the circuit.  


One general approach is to construct statistical models reflecting hypothesized models of neural circuit function, including temporal dynamics, nonlinear interactions between neurons, and background network activity \citep{Pillow:2008,Buonomano:2009,Yu:2009}. 
Another general approach is to search for alternative statistics that best describe the population activity without assuming a specific model.  Such evaluations rely on constructing surrogate datasets \citep{Okun:2012} or maximum-entropy distributions that reproduce the statistic of interest \citep{Ganmor:2011,Tkacik:2012} with subsequent testing of how well such constructs can reproduce other observed properties of population activity.

Such more sophisticated statistical descriptions are unlikely to entirely replace neural correlations as the initial descriptive statistic of population activity in experimental neuroscience. Neural correlations have several strengths that will ensure their continued prominence: (a) correlations are well established, familiar, and intuitive to most researchers in biological sciences, (b) correlations are easy to estimate as they are less susceptible to the curse of dimensionality than more sophisticated models, (c) correlations make relatively weak mechanistic assumptions (at the cost of not revealing many potentially important aspects), and (d) correlations serve as input into other models of population activity or decoding algorithms. 

Linear correlations between pairs of neurons are among the most common and familiar descriptions of functional associations in networks and of the collective activity of neuronal populations.  For example, it is tempting to consider the network constructed from the highest correlations in the recorded population (e.g.~\cite{Malmersjo:2013}) for subsequent graph-theoretical analysis \citep{Feldt:2011}.  In functional genomics, networks constructed from highest correlations are called \emph{relevance networks}.

However, pairwise correlations unreliable proxies of direct functional association such as direct synaptic connectivity as they can arise due to a wide variety of physiological interactions including direct synaptic interactions, indirect chains of synaptic connections, common inputs, correlations in inputs, synchrony, fluctuations in global network states, and others \citep{Shadlen:1998,Ostojic:2009,Pernice:2011,Schneidman:2006}.

Other experimental evidence suggests that detailed interactions in local microcircuits have only weak effects on overall population activity, which is then best characterized by collective features such as global population dynamics or population synchrony \citep{Okun:2012,Tkacik:2012,Tkacik:2013}.  From this point of view, the imporant aspects of the correlation structure lie it its global aspects, such as the eigenspectrum, while the individual pairwise correlations are not particularly meaningful.


\section*{Acknowledgement}


\paragraph{Funding\textcolon}

\section*{Supplemental Data}


\bibliographystyle{frontiersinSCNS&ENG} % for Science and Engineering articles
\bibliography{references.bib}

\end{document}
