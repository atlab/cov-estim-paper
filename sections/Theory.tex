\subsection*{Basic concepts}
Let $x(t) \in \mathcal X, t=1\ldots,n$ denote a sample of consecutive observations of population activity of $p$ neurons in time bins $t$.  
The activity of each neuron is denoted by the real-valued firing rate, thus  $\mathcal X = \mathbb R^{p\times 1}$.  
For example, in our application, the firing rate is estimated from somatic calcium fluorescence signals with the mean stimulus response subtracted, making it possible for elements of $x(t)$ to take on negative and non-integer values. 

We do not assume that the observations $x(t)$ are independently distributed. Rather, the data generating process is assumed to be \emph{ergodic}, \emph{i.e.}\;described by a \emph{true distribution} $F: \mathcal X \mapsto  \mathbb [0, 1]$ (cumulative) over long periods of time such that the \emph{empirical distribution} $\hat F_n$ from a given sample will convege to $F$ with increasing sample size $n$.

Formally, the emprical distribution is defined as \TODO{this may be unecessary, but I leave it for now.}
\begin{equation}
\hat F_n(x) = \frac 1 n \sum\limits_{t=1}^n \mathbf{1}(x \ge x(t))
\end{equation}
where $\mathbf 1(x \ge x(t))$ is the indicator function which equals 1 when all elements of $x$ are greater than the corresponding elements of $x(t)$ and 0 otherwise. \TODO{KJ: Are you using cumulative distributions to avoid binning? DY: Yes. Many theorems such as Glivenko-Catelli Theorem and Skorohod's Representation are proven using  cumulative representation. The empirical distribution is nearly always represented in its cumulative form. Expressing convergence is requires binning or using cumulative; the latter is simpler.}

For an ergodic process, $\max\limits_{x\in\mathcal X} \left|\hat F_n(x) - F(x)\right| \to 0$ as $n$ increases.  \TODO{If you decide to leave this in, provide a reference to a standard book in the field.}

The true covariance matrix $\Sigma \in \Theta$ is defined as a function of $F$:
\begin{equation}
\Sigma = \int\limits_{z\in\mathcal X} (z - \mu)(z - \mu)^\T \dif F(z)
\quad\mbox{where}\;
\mu = \int\limits_{z\in\mathcal X} z \dif F(z)
\end{equation}
The domain $\Theta$ is the set of all positive-definite $p\times p$ matrices, which is a cone in $\mathbb R^{p\times(p+1)/2}$.

But the usual estimator of the true covariance matrix $\Sigma$ is  the sample covariance matrix:
\begin{equation}
\hat\Sigma_0 =  \frac 1 {n-c} \sum\limits_{t=1}^n (x(t) -\hat\mu) (x(t) - \hat\mu)^\T
\quad\mbox{where}\;
\hat\mu = \frac 1 n \sum\limits_{t=1}^n x(t)
\end{equation}
where $c$ is the sum of temporal correlations \TODO{there must be a better description of this} \TODO{Which?  I think the equation is OK.  Do you mean $c$?  This is standard, and again, I suggest just providing a reference.}. For independent observations, $c=1$;  $c>1$ when nearby samples are correlated; the value of $c$ can be estimated from the data.
\TODO{Under the Gaussian loss, $\mathcal L_g$ (\ref{eq:GaussLoss}) the scaling is irrelevant, so we will not spend any time talking about the scaling.  But it would be important if we used $\mathcal L_e$ (\ref{eq:MSE}).} 

$\hat\Sigma_0$ is designed to be unbiased such that $\mathbb E\left[\hat\Sigma_0\right]=\Sigma$.
Here and througout, $\mathbb E[\cdot]$ denotes the expected value under the  unknown true distribution $F$. All variables with a hat (\emph{e.g.}\;$\hat \Sigma_0,\hat \mu$) are functions of the empirical distribution $\hat F_n$, which itself is a random variable.

%Nonlinear functions of an unbiased estimate can be biased: $\mathbb E\left[\hat\Sigma_0\right]=\Sigma \centernot\implies  E\left[\varphi(\hat\Sigma_0)\right]=\varphi(\Sigma)$.  For example, if $u_{\max}(S)$ is the largest eigevalue of square matrix $S$, then $\mathbb E\left[u_{\max}(\hat\Sigma_0)\right] > u_{\max}(\Sigma)$. Some covariance matrix estimators are designed to correct for the eigenspectrum bias at the cost  of adding bias to covariance coefficients \cite{Ledoit:2004}.




\subsection*{Loss and risk}
The optimization of a covariance matrix estimate $\hat\Sigma$ is performed with respect to a \emph{loss function} $\mathcal L(\hat\Sigma,\Sigma)$, which expresses the discrepancy between $\hat\Sigma$ and $\Sigma$ and attains its minimum when $\hat\Sigma=\Sigma$.  
Then \emph{excess loss}  
\begin{equation}
\ell(\hat\Sigma,\Sigma) = \mathcal L(\hat\Sigma,\Sigma)-\mathcal L(\Sigma,\Sigma)
\end{equation}
assumes zero at its minimum.

A particularly useful loss function is the mean squared error (MSE), which is proportional to the square of the Frobenius  norm $\|\cdot\|_F$ of the difference between the two matrices: 
\begin{equation}\label{eq:MSE}
\mathcal L_e(\hat\Sigma,\Sigma) =\frac 1 p \|\hat\Sigma-\Sigma\|_F^2 = \frac 1 p \Tr\left((\hat \Sigma-\Sigma)(\hat\Sigma-\Sigma)^\T\right)
\end{equation}
Since $\mathcal L_e(\Sigma,\Sigma) \equiv 0$, the MSE is its own excess loss: $\ell_e(\hat\Sigma,\Sigma) \equiv \mathcal L_e(\hat\Sigma,\Sigma)$.

The Gaussian loss function $\mathcal L_g$  arises from the theory of multivariate normal distributions. When observations are identically and independently distributed according to a multivariate normal distribution with zero means, the log likelihood of the covariance matrix $\Sigma$ with $\hat\Sigma = \frac 1 n \sum\limits_{t=1}^n x(t) x(t)^\T$ is  
\begin{equation}
L\left(\Sigma \mid \hat\Sigma\right) = -\frac n 2 \ln(2\pi) - \frac n 2 \ln \det \Sigma - \frac n 2 \Tr(\Sigma^{-1} \hat \Sigma)
\end{equation}
Then $\mathcal L_g$ is constructed by rescaling $L\left(\Sigma \mid \hat\Sigma\right)$ and dropping the constant term:
\begin{equation}\label{eq:GaussLoss}
\mathcal L_g(\hat\Sigma,\Sigma) 
=  -\frac 2 {pn} L\left(\hat\Sigma \mid \Sigma \right) - \frac 1 p \ln(2\pi) 
\equiv  \frac 1 p\left(\ln \det \hat \Sigma + \Tr(\hat \Sigma^{-1}\Sigma) \right) 
\end{equation}
\TODO{Note that the switch from $L\left(\Sigma \mid \hat\Sigma \right)$ to $L\left(\hat\Sigma \mid \Sigma \right)$ is on purpose, which will become clear in the next few steps.}
The corresponding excess loss 
\begin{equation}
\ell_g(\hat\Sigma,\Sigma) = \mathcal L_g(\hat\Sigma,\Sigma) - \mathcal L_g(\Sigma,\Sigma)  
= \frac 1 p \left(-\ln \det (\hat \Sigma^{-1} \Sigma) + \Tr(\hat \Sigma^{-1}\Sigma)\right) - 1
\end{equation}
is known as \emph{entropy loss} \cite{James:1961}. \TODO{The order of the operands is reversed. Check that it's okay.}

Despite the fact that entropy loss is derived from normal theory, the choice of a loss function is not equivalent to assuming a specific form of $F$. The loss function measures the discrepancy between distribution parameters rather than the distance between the distributions themselves. \TODO{explain Bregman divergence?} 

The expected value of excess loss is the \emph{estimator risk}:
\begin{equation}\label{eq:risk}
r = \mathbb E\left[\ell(\hat\Sigma,\Sigma)\right]
\end{equation}
The estimator risk is the primary quality criterion for covariance estimation. Estimator $\hat\Sigma_a$ is considered more \emph{efficient} than estimator $\hat\Sigma_b$ if $\hat\Sigma_a$ has lower risk for the given sample than $\hat\Sigma_b$.   \TODO{explain that the risk depends not only on $\hat\Sigma$ but also on the distribution of $\Sigma$ in the specific domain.}

\subsection*{Cross validation}
In practice, the true covariance matrix $\Sigma$ is not accessible and the estimator risk $r$ must be estimated from data. This requires a separate \emph{validation} empirical distribution $\hat F_m^\prime$ of $m$ observations sampled from $F$ independently of the \emph{training} distribution $\hat F_n$. 
Then let  $\hat \Sigma_0^\prime$ be the sample covariance obtained from $\hat F_m^\prime$. 

Then the \emph{empirical loss} of $\hat\Sigma$ is $\mathcal L(\hat\Sigma,\hat\Sigma_0)$ and its expectation is the \emph{empirical risk}  
\begin{equation}
\hat r = \mathbb E\left[\mathcal L(\hat\Sigma,\hat\Sigma_0^\prime)\right]
\end{equation}

 The two loss functions $\mathcal L_e(\hat\Sigma,\Sigma)$ and $\mathcal L_g(\hat\Sigma,\Sigma)$ are particularly suitable for risk estimation thanks to their linearity with respect to $\Sigma$ in the sense that 
\begin{equation}
\mathcal L\left(\hat\Sigma,\alpha S_1 + (1-\alpha)S_2\right) 
\equiv 
\alpha\mathcal L(\hat \Sigma,S_1) + (1-\alpha)\mathcal L(\hat \Sigma,S_2)
\end{equation}
which allows bringing the expectation inside the empirical loss function:
\begin{equation}
\hat r = 
\mathbb E\left[ \mathcal L(\hat\Sigma, \hat\Sigma_0^\prime) \right] 
=
\mathbb E\left[ \mathcal L\left(\hat\Sigma, \mathbb E\left[\hat\Sigma_0^\prime\right]\right) \right] 
=
\mathbb E\left[ \mathcal L(\hat\Sigma, \Sigma) \right] 
= 
r + \mathcal L(\Sigma,\Sigma)
\end{equation}
This means, that the empirical loss $\mathcal L(\hat\Sigma,\hat\Sigma_0^\prime)$ is an unbiased estimate of the estimator risk $r$ (up to a constant offset): minimization of the empirical loss implies minimization of the true estimator risk. 

Recording a separate independent validation sample is usually not sensible.  Instead, the recorded data are split into a training sample and validation sample. In $K$-fold \emph{cross-validation} the data are split into $K$ roughly equal-sized subsets. Each subset then successively serve as the testing sample while the remainder of the data serves as the training sample.  This procedure produces $K$ estimates of the  risk, which are then averaged together to produce a better averaged estimate.


\subsection*{Bias/variance decomposition}
The estimator risk (\ref{eq:risk}) can be decomposed into \emph{approximation error} or \emph{bias}   
\begin{equation}
b^2 = \ell \left( \mathbb E[\hat \Sigma],\Sigma\right)
\end{equation}
and \emph{estimation error} or \emph{variance}
\begin{equation}
\varepsilon^2 = \mathbb E \left[ \ell\left(\hat \Sigma, 
\mathbb E[\hat \Sigma]\right) \right]
\end{equation}

Under the MSE loss function $\mathcal L_e$ (\ref{eq:MSE}), the decomposition is the simple sum $r = b^2 + \varepsilon^2$. 
Under other loss functions, $r$ is an increasing function of both $b^2$ and $\varepsilon^2$ although not generally a simple sum.  
\TODO{As far as I know, the terms `bias' and `variance' are only applicable under a quadratic loss function  like $\mathcal L_e$.  Under other loss functions, we may need to use approximation/estimation error.} 

The sample covariance matrix $\hat\Sigma_0$ has zero bias but high variance. Other estimators may be biased but have lower estimation error and potentially lower risk.
