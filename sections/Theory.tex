\subsection{Covariance estimation}

\paragraph{Problem setup:}

Let $x_i \in \mathcal X, i=1\ldots,x_n$ denote a sample of consequtive observations of population activity of $p$ neurons in time bins $i$.  
The activity of each neuron is the real-valued firing rate, thus  $\mathcal X = \mathbb R^{p\times 1}$.  
We assume that observations are drawn from some true cumulative probability distribution $F: \mathcal X \mapsto R$. Then let $\hat F_n$ denote the empirical distribution function obtained from the observed sample, which itself is a random variable.
\begin{equation}
\hat F_n(x) = \frac 1 n \sum\limits_{i=1}^n \mathbf{1}(x \ge x_i)
\end{equation}
where $\mathbf 1(x \ge x_i)$ is the indicator function which equals 1 when all elements of $x$ are greater than the corresponding elements of $x_i$ and 0 otherwise.

The observations need not be independently and identically sampled from $F$. The only assumption is that the data generating process is ergodic, \emph{i.e.}\;that $\max_x \left[\hat F_n(x) - F(x)\right] \to 0$ as the sample size $n$ increases. Then each $x_i$ can be thought to be drawn from $F$ when not conditioned on nearby observations.

The true covariance matrix $\Sigma \in \Theta$ is defined as a function of the true probability distribution
\begin{equation}
\Sigma = \int\limits_{z\in\mathcal X} (z - \bar x)(z - \bar x)^\T \dif F(z)
\quad\mbox{where}\;
\bar x = \int\limits_{z\in\mathcal X} z \dif F(z)
\end{equation}
The \emph{plug-in estimator} $\hat\Sigma$ is obtained by plugging the empirical distribution $F_n$ in place of the true distribution $F$
\begin{equation}
\begin{split}
\hat\Sigma = \int\limits_{z\in\mathcal X} (z - \bar x)(z - \bar x)^\T \dif \hat F_n(z) 
\equiv \frac  1 n\sum\limits_{i=1}^n (x_i-\bar x)(x_i - \bar x)
\\
\mbox{where}\;\bar x = 
\int\limits_{z\in \mathcal X}z \dif \hat F_n(z) \equiv \frac 1 n\sum\limits_{i=1}^n x_i
\end{split}
\end{equation}
The plug-in estimator is known to be biased toward zero because the estimate of the mean $\bar x$ absorbs a fraction of the variances. 

The usual unbiased estimate, the sample covariance matrix, rescales the plug-in estimator to compensate for the bias
\begin{equation}
\hat\Sigma_0 = \frac n {n - d} \hat\Sigma \equiv \frac 1 {n-d} \sum\limits_{i=1}^n (x_i - \bar x) (x_i - \bar x)^\T
\end{equation}
When observations are independent, then $d=1$.  But when nearby samples are correlated, the estimate of the mean absorbs more of the variance, becomes \emph{overdispersed}. As a result the covariances become underestimated. Then we can use $d=cn$ where $c>1$ is called the \emph{overdispersion coefficient}, which may be estimated from the data.

\paragraph{Variance-bias tradeoff:}
Let $\hat\Sigma_\gamma$ be a family of approximate (biased) covariance estimates indexed by $\gamma$. To evaluate each estimate, let's define the \emph{loss function} $\mathcal L : \Theta \times \Theta \mapsto \mathbb R$ such that $\mathcal L(\hat S,S)$ is minimized when $\hat S = S$.  Then \emph{excess loss} $\ell(\hat S, S) = \mathcal L(\hat S, S) - \mathcal L(S,S)$ assumes zero at the minimum.

The overall error or \emph{estimator risk} 
\begin{equation}
r_\gamma = \mathbb E\left[ \ell\left(\hat\Sigma_\gamma,\Sigma\right) \right]
\end{equation}
can be decomposed into \emph{approximation error} or \emph{bias}   
\begin{equation}
b_\gamma^2 = \ell \left( \mathbb E\left[\hat \Sigma_\gamma\right],\Sigma\right)
\end{equation}
and \emph{estimation error} or \emph{variance}
\begin{equation}
\varepsilon_\gamma^2 = \mathbb E \left[ \ell\left(\hat \Sigma_\gamma, 
\mathbb E\left[\hat \Sigma_\gamma\right]\right) \right]
\end{equation}

When the loss function is the error sum of squares, which, for matrices, is called the Frobenius norm, 
\begin{equation}
\mathcal L_e(\hat S,S) = \|\hat S-S\|_F^2 = \Tr\left((\hat S-S)(\hat S-S)^\T\right)
\end{equation}
the bias-variance decomposition is the simple sum
\begin{equation}
r_\gamma =  b_\gamma^2 + \varepsilon_\gamma^2
\end{equation}

With other loss functions, the risk is also an increasing function of the two components, although is not generally the sum.

Regularization is the game of minimizing risk $r_\gamma$ by reducing estimation error $\varepsilon_\gamma$ at the cost of increased bias $b_\gamma$, the \emph{bias-variance tradeoff}.  
We follow the convention that $\hat\Sigma_{\gamma=0}$ is the unbiased estimator $\hat\Sigma_0$, \emph{i.e.}\;$b_{\gamma=0}^2 = 0$.  Then as $\gamma$ increases, $\hat\Sigma_\gamma$ less variable but increasingly biased.  Thus $\hat\Sigma_\gamma$ defines the \emph{regularization path}, for which \emph{regularization intensity} $\gamma$ specifies the location along the path, which may be real-valued or discrete, or even a vector to steer the bias toward multiple targets.  

\paragraph{Evaluation of estimators:}
Another popular loss  function for covariance estimation arises from the theory of normal distributions.  Under the assumption of normality  with zero mean, the log likelihood of the covariance matrix $\Sigma$ given the independent sample $z_1,\ldots,z_m$, with $S = \tfrac 1 m \sum\limits_{i=1}^m z_i z_i^\T$,  
\begin{equation}
L\left(\Sigma \mid S\right) = -\tfrac m 2 \ln(2\pi) - \tfrac m 2 \ln \det \Sigma - \tfrac m 2 \Tr(\Sigma^{-1} S)
\end{equation}
Then the Gaussian loss function $\mathcal L_g$ is constructed by rescaling the normal log likelihood and dropping the constant term
\begin{equation}
\begin{split}
\mathcal L_g(\hat S,S) = &-\frac 2 m L\left(S \mid \hat S \right) - \ln(2\pi) 
\\ 
\equiv & \ln \det \hat S + \Tr(\hat S^{-1} S) 
\end{split}
\end{equation}
The corresponding excess loss 
\begin{equation}
\ell_g(\hat S,S) = \mathcal L_g(\hat S,S) - \mathcal L_g(S,S) 
= -\ln \det \left(\hat S^{-1} S\right) + \Tr\left(\hat S^{-1}S\right) - p
\end{equation}
is known as \emph{entropy loss} \citep{James:1961}.

Despite the fact that entropy loss is derived from normal theory, the choice of a loss function is not equivalent to assuming a specific form of $F$. The loss function measures the discrepancy between distribution parameters rather than the distance between distributions. \TODO{explain Bregman divergence?} 

In practice, the true covariance matrix $\Sigma$ is not available and the estimator risk must be estimated from data. This requires a separate \emph{validation} empirical distribution $\hat F_m^\prime$ of $m$ observations sampled from $F$ independently of the \emph{training} distribution $\hat F_n$. The we can produce an independent unbiased covariance estimate $\hat \Sigma_0(\hat F_m^\prime)$.

Then the \emph{empirical risk} is 
\begin{equation}
\hat r_\gamma = \mathcal L\left(\hat\Sigma_\gamma,\hat\Sigma_0(\hat F_m^\prime)\right) 
\end{equation}

 The two loss functions $\mathcal L_e(\hat S,S)$ and $\mathcal L_g(\hat S,S)$ are particularly suitable for empirical risk estimation thanks to their linearity with respect to $S$ in the sense that 
\begin{equation}
\mathcal L\left(\hat S,\alpha S_1 + (1-\alpha)S_2\right) 
\equiv 
\alpha\mathcal L(\hat S,S_1) + (1-\alpha)\mathcal L(\hat S,S_2)
\end{equation}
which allows bringing the expectation inside 
\begin{equation}
\mathbb E\left[ \mathcal L\left(\hat\Sigma_r, \hat\Sigma_0(\hat F_m^\prime)\right) \right] 
=
\mathbb E\left[ \mathcal L\left(\hat\Sigma_\gamma, \mathbb E\left[\Sigma_0(\hat F_m^\prime)\right]\right) \right] 
=
\mathbb E\left[ \mathcal L\left(\hat\Sigma_\gamma, \Sigma\right) \right] 
\end{equation}
Then the expectation of the empirical risk is 
\begin{equation}
\mathbb E\left[\hat r_\gamma\right] 
= \mathbb E\left[\mathcal L(\hat\Sigma_\gamma,\Sigma)\right]
= r_\gamma + \mathcal L(\Sigma,\Sigma)
\end{equation}
This means, that our goal of minimizing the true loss $r_\gamma$ can be replaced with minimizing  the empirical loss $\hat r_\gamma$.

In practice, recording a separate independent validation sample may not be sensible.  Instead, the recorded data a split into a training sample and validation sample. In $K$-fold \emph{cross-validation} the data are split into $K$ roughly equal-sized subsets. Each subset then successively serves to construct the validation distribution $\hat F_m^\prime$  while the rest of the data is used to construct the training distribution $\hat F_n$.  This procedure produces $K$ estimates of the empirical risk, which are then averaged together to produce a better, averaged, estimate of the true risk.

Different regularization schemes can then be compared by comparing cross-validated empirical losses.

 
\paragraph{Constructing a regularized estimator:}
The task of designing a regularized estimators consists of defining a sequence of approximations $\hat\Sigma_\gamma$, called \emph{regularization path} and the optimal location on that path, \emph{regularization intensity} $\hat\gamma$ as  functions of the empirical distribution. 

There are two basic approaches to design a family of approximations of a parameter that reduce the estimation error: \emph{dimensionality reduction} and \emph{shrinkage}.  

Dimensionality reduction projects the estimate onto a low-dimensional manifold (subspace) within the $p(p+1)/2$-dimensional space of covariance matrices $\Theta$.  The variance in the projection is likely to be lower (if the manifold is smooth).  For example, factor analysis with $\nu$ factors projects (w.r.t.\;$\mathcal L_g$) $\hat\Sigma_0$ onto the low-dimensional manifold of covariance matrices representable by factor models. Thus factor analysis can be used as a regularized covariance estimator \citep{Fan:2008} with $\gamma$ controlling the number of factors (or, alternatively, the nuclear norm).  Another dimensionality reduction technique is \emph{covariance selection} \citep{Dempster:1972}, wherein a fraction of inverse covariance coefficients are constrained to zero and the closest (w.r.t.\;$\mathcal L_g$) solution to $\hat\Sigma_0$ is found. 

Covariance shrinkage \citep{Schafer:2005,Ledoit:2004} uses a fixed, low-dimensional \emph{target estimate} $T$ and constructs the estimator as the linear mixture, with $\gamma \in [0,1]$,
\begin{equation}
\hat\Sigma_\gamma = \hat\Sigma_0  + \gamma (T-\hat\Sigma_0) 
\end{equation}
 The target can be a diagonal matrix with sample variances, an identity matrix scaled to the mean variance, a single factor model, or any other low-dimensional estimate.

An effective hybrid of dimensionality reduction and shrinkage can be obtained by using $L_1$-regularization techniques.  For example, graphical lasso \citep{Meinshausen:2006,Friedman:2008}, shrinks the inverse of the covariance toward zero as 
\begin{equation}
\hat\Sigma_\gamma = \arg\min\limits_S \mathcal L_g(S,\hat \Sigma_0) + \gamma \| S^{-1} \|_1
\end{equation}
This approach results in forcing many of the coefficients in the inverse of the estimate to zero, or ``sparsification'' of the the inverse.

Many regularization schemes combine multiple targets or multiple approaches, in which case $\gamma$ can become a vector of regularization parameters \citep{Schafer:2005,Fan:2011,Ma:2013} to be optimized in parallel.

\paragraph{Optimization of the regularization parameter:}
Analytic solutions when exist \citep{Ledoit:2004,Schafer:2005}, nested cross-validation otherwise.

\paragraph{Interpretation of revealed structure:}
Sparseness (dimensionality reduction) are key, but must be supported by data, not emph{ad hoc}. \citep{Fan:2006,Malmersjo:2013}.

\paragraph{Out choices of regularized covariance estimates:}
\begin{enumerate}[\qquad 1.\;\;]
\item sample covariance (benchmark)
\item linear shrinkage toward identity 
\item linear shrinkage toward a single-factor model.
\item multifactor
\item graphical lasso
\item latent-variable graphical lasso
\end{enumerate}

