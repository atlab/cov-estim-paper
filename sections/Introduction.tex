\hl{\tiny why correlations:}
Linear correlations between the spiking activity of pairs of neurons, or simply \emph{neural correlations}, are among the most familiar descriptive statistics of neural population activity \citep{Cohen:2011}.   
For example, \emph{noise correlations}, i.e.~the correlations of stimulus response variability between pairs of neurons in sensory areas, have been of particular interest thanks, in part, to their profound theoretical implications for stimulus coding \citep{Zohary:1994,Abbott:1999,Averbeck:2006,Berens:2011}.
Lending further credence to their role as indicators of detailed and specific functional organization have been a series of discoveries of nontrivial relationships between neural correlations and other aspects of circuit organization such as the physical distance separating the neurons, their synaptic connectivity and stimulus tuning similarity, cortical layer specificity, cell-type specificity, progressive changes in development and in learning, changes due to sensory stimulation and global brain states, and others \citep{Kohn:2005,Smith:2008,Kohn:2009,Goard:2009,Golshani:2009,Renart:2010,Ecker:2010,Smith:2013,Denman:2013}. 

\hl{\tiny ambiguity of interpretation:}
Neural correlations do not come with ready or unambiguous mechanistic interpretations.   
Theoretical and simulation studies show that neural correlations at various temporal scales may arise from any combination of underlying mechanisms such as direct synaptic interactions,  common or correlated inputs, shared sensory noise, chains of multiple synaptic connections, oscillations, top-down modulation, and background network activity \citep{Perkel:1967b,Shadlen:1998,Salinas:2001,Ostojic:2009,Rosenbaum:2011}

\hl{\tiny opportunities with multineuronal data:}
Early studies of neural correlations were based on measurements from isolated pairs of neurons and their impact on coding was extrapolated to entire populations \citep{Shadlen:1998,Zohary:1994}.  Modern massively multineuronal recordings allow estimations of entire correlation matrices from large neuronal populations, which provide more information than the equivalent number of pairwise correlations assessed in isolation.
After all, the correlation matrix is greater than the sum of its parts: it can be transformed into other representations  that accentuate different aspects of its structure and may suggest different mechanistic interpretations.  
For example, the eigenvalue decomposition, also known as the principal component analysis, of the covariance matrix expresses shared correlated activity components across the entire population;  common fluctuations of population activity of the entire circuit may be accurately represented by just a few principal components but will affect all correlation coefficients. 
In contrast, the off-diagonal elements of the inverse of the correlation matrix constitute scaled partial correlations between neuron pairs, which reflect their specific linear dependencies, after accounting for all other recorded cells;
a strong interaction between a pair of neurons may be expressly represented by a single partial correlation but its effects propagate to multiple correlations and principal components.

\hl{\tiny motivate regularization} Estimation of parameters of the covariance matrix from data is inherently challenging, increasingly for very high-dimensional data. As the amount of data grows only linearly with population size, the number of free parameters in a complete representation of the covariance matrix increases quadratically, multiplying opportunities for spurious patterns to emerge among the correlations. 
Although the estimation error of each covariance coefficient, in isolation, is unaffected by the dimensionality of the data, the estimation error of other parameters of the covariance matrix increases with population size.  
For example, the large eigenvalues of in the sample covariance matrix are biased upward while small eigenvalues are biased downward, with both variance and bias increasingly large with increasing data dimensionality \citep{Hayes:1981}.  Similarly, the coefficients of the inverse covariance matrix require  larger sample sizes to be estimated accurately in high-dimensional data.

\hl{\tiny forms of regularization} The convergence of the estimate of multivariate parameters to their true values can be improved through \emph{regularization},  Regularization is the deliberate biasing or \emph{shrinkage} of the empirical estimate toward a less variable \emph{target estimate} \citep{Schafer:2005} or the deliberate restriction of the estimate to a lower-dimensional space \citep{Dempster:1972,Fan:2008,Friedman:2008,Rothman:2008}. 
\hl{\tiny The two definitions overlap and are often equivalent, but not always, according to James, but I would like to see an example of shrinkage that is not, effectively, a low dimensional parameterization of the original problem.} 
One intuitive explanation of regularization is as the addition of some  prior knowledge about the likely forms of the covariance matrix. Curiously, though, regularization need not reflect any accurate knowledge about the nature of covariance matrices in general or in the particular domain. Some improvement arises due to \emph{Stein's phenomenon}, i.e. the favorable tradeoff between bias and variability \emph{anytime} a moderate bias toward a less variable target is introduced.  
However, the incorporation into the regularization scheme of real knowledge about the probable forms of covariance matrices for a specific situation will likely confer additional advantage and improve the convergence more substantially than an \emph{ad hoc} regularizer.

%Estimator improvements due to regularization depend profoundly on the parameterization to which it is applied.  Sparsification of the eigenspectrum produces a low-rank estimate of the covariance matrix also known as truncated PCA \citep{Rothman:2008} or, allowing for additional independent variances, as factor analysis (FA) \citep{Fan:2008}.  Sparsification of the inverse covariance is known as covariance selection \citep{Dempster:1972,Friedman:2008} and is related to finding a graphical model in which a large fraction of neuron pairs are conditionally independent of each other.  Representations that are capable of concentrating the significant real effects in a small number of parameters will reap greatest benefits from sparsification over ones that distribute the real effects widelyacross many parameters.  Random noise tends to be broadly distributed across all parameters in any parameterization. 

Two questions must be answered in the choice of the regularization scheme: what kind of regularization and how much regularization. The answer to the first question is domain-specific and is often motivated by theories of the origins of correlations in the specific subject domain.  


\hl{\tiny approach and summary of findings} 
In this study, we compared the performance of several covariance estimation schemes for spatially compact groups of 150--300 neurons in layers 2-4 in mouse primary visual cortex during visual stimulation.   The performance of the estimators was evaluted by computing the mean squared error between the optimized covariance estimate fitted to training data and the non-regularized covariance matrix from a separate test data set.  Low-rank covariance estimates performed significantly better than shrinkage estimators, but estimators with sparse partial correlations were more efficient still. Typically, between 3 and 16\% of neuronal paris were connected by non-zero partial interactions.  Mixed sparse estimators with sparse partial correlations and low-rank components performed comparably. 

As noted above, the optimal covariance estimator is domain-dependent. 
These specific findings may not spread to all neural circuits.
Improved accuracy estimation accuracy is not the main aim or the only benefit.  
The selection of the optimal regularization scheme in itself serves as an effective descriptor of the dominant correlation structures. The finding that a identifiable subsets of partial correlations were important to describe the partial correlation makes such pairs as prime candidates as direct functional interactions. 