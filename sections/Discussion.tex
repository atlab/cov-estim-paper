\TODO{The discussion section is not yet ready for feedback. Currently it's a collection of scraps from earlier versions of the introduction.}

Fundamental questions in systems neuroscience concern the relationship between the function of neural microcircuits and their cytoarchitecture: circuit topology, cell types, and patterns of synaptic connectivity. 

Considerable progress has been made in sensory areas where the functional characterizations of cells are defined by their reproducible responses to external stimuli (reviewed in \citep{Reid:2012}). 

Many key questions in neuroscience revolve around the relationship between the cytoarchitecture of neural microcircuits and the functional organization of their activity \citep{Reid:2012}.  Investigators in this line of inquiry aim to construct networks of functional associations within groups of neurons inferred from observations of their activity under a variety of conditions.  The inferred functional network  could then related to the synaptic connectivity patterns, cell types, cell properties, or the cells' spatial arrangment in the attempt to uncover organizational principles of neural computation. In addition, the network itself can serve as subtrate for subsequent graph-theoretical analysis \citep{Feldt:2011}

Advances in electrophysiology and optical imaging have enabled simultaneous recordings from dozens to hundreds or thousands of cells. 
In particular, recent advances in two-photon imaging of calcium signals have allowed in vivo recordings from nearly every neuron brain-wide in zebrafish \citep{Leung:2013,Ahrens:2013} and in a 3D volume of a few hundred microns in diameter in mouse neocortex \citep{Katona:2012,Cotton:2013}.    
Ambitious projects currently under way aim to record the spiking activity of large fractions of cells from entire circuits and systems in behaving animals \citep{Alivisatos:2012}.  Direct observations of the population activity  of entire circuits open new possibilities for incisive statistical descriptions of the functional connectivity in the circuit.  

One general approach is to construct statistical models reflecting hypothesized models of neural circuit function, including temporal dynamics, nonlinear interactions between neurons, and background network activity \citep{Pillow:2008,Buonomano:2009,Yu:2009}. 
Another general approach is to search for alternative statistics that best describe the population activity without assuming a specific model.  Such evaluations rely on constructing surrogate datasets \citep{Okun:2012} or maximum-entropy distributions that reproduce the statistic of interest \citep{Ganmor:2011,Tkacik:2012} with subsequent testing of how well such constructs can reproduce other observed properties of population activity.

Such more sophisticated statistical descriptions are unlikely to entirely replace neural correlations as the initial descriptive statistic of population activity in experimental neuroscience. Neural correlations have several strengths that will ensure their continued prominence: (a) correlations are well established, familiar, and intuitive to most researchers in biological sciences, (b) correlations are easy to estimate as they are less susceptible to the curse of dimensionality than more sophisticated models, (c) correlations make relatively weak mechanistic assumptions (at the cost of not revealing many potentially important aspects), and (d) correlations serve as input into other models of population activity or decoding algorithms. 

Linear correlations between pairs of neurons are among the most common and familiar descriptions of functional associations in networks and of the collective activity of neuronal populations.  For example, it is tempting to consider the network constructed from the highest correlations in the recorded population (e.g.~\cite{Malmersjo:2013}) for subsequent graph-theoretical analysis \citep{Feldt:2011}.  In functional genomics, networks constructed from highest correlations are called \emph{relevance networks}.

However, pairwise correlations unreliable proxies of direct functional association such as direct synaptic connectivity as they can arise due to a wide variety of physiological interactions including direct synaptic interactions, indirect chains of synaptic connections, common inputs, correlations in inputs, synchrony, fluctuations in global network states, and others \citep{Shadlen:1998,Ostojic:2009,Pernice:2011,Schneidman:2006}.

Other experimental evidence suggests that detailed interactions in local microcircuits have only weak effects on overall population activity, which is then best characterized by collective features such as global population dynamics or population synchrony \citep{Okun:2012,Tkacik:2012,Tkacik:2013}.  From this point of view, the imporant aspects of the correlation structure lie it its global aspects, such as the eigenspectrum, while the individual pairwise correlations are not particularly meaningful.
