Linear correlations between the spiking activity of pairs of neurons are among the most familiar and useful descriptive statistics of neural activity.
Correlation matrices from multineuronal recordings contain richer information than available from an equivalent number of isolated pairs:
The basic structure of population activity can manifest in various low-dimensional representations of the covariance matrix.  \Kcomment{The previous may not be clear to someone who has not had experience with this before.  What low-dimensional representations? What basic structures?  Maybe follow this sentence with an example?  I think being more concrete, even at the expense of generality, will bring the point across more clearly.  You give examples below, so maybe just reorder.}  The optimal choice of such reduced representation  depends profoundly on how well it can capture regularities in the underlying data generating process. \Kcomment{Which reduced representation?  I understand that the two sentences are related, but ``low-dimensional representation'' and ``reduction'' are not necessarily equivalent.}
For example, diffuse correlated input onto all neurons implies a low-rank parameterization such as used in  factor analysis whereas presumed strong pairwise linear associations imply sparse partial correlations.  \Kcomment{I am no sure that the second part is true.  Couldn't we have strong all-to-all pairwise correlations without sparse partial correlations?  }
Composite  `sparse+low-rank partial correlation' parameterizations allow combining common correlated input with strong pairwise associations.  \Kcomment{This last sentence is a jump to a new idea (combining the two structures), which is important, but just is a hanger on here.}

To infer the optimal low-dimensional representation of neural correlations in a specitifc neural microcircuit, we compared the performance of several estimators of the covariance matrix on the activity of 100--300 neurons in mouse visual cortex: sample covariance, covariance shrinkage, factor analysis, sparse partial correlations, and sparse+low-rank partial correlations. 
The instantaneous firing rates, binned at 200 ms, were inferred from the somatic calcium signals acquired with fast 3D random-access two-photon microscopy.   
As expected, covariance shrinkage reliably outperformed the sample covariance estimate. \Kcomment{Why ``as expected''?  }
Factor analysis-based estimates significantly outperformed covariance shrinkage. \Kcomment{This was not completely expected.  Do you want to say something else?}
Yet sparsified partial correlations with or without an additional low-rank component significantly outperformed both factor analysis and shrinkage-based  estimators. \Kcomment{Is it clear where this approach fits in the list you give above?}
The superior performance of sparse partial correlation-based estimates suggests the relative importance of detailed network interactions over common input in this circuit.
The resulting sparse partial correlation matrix, we argue, serves as a better substrate for functional network analysis. 
In addition, the overall estimation improvement will propagate to any existing analyses that rely on the covariance matrix.
\Kcomment{Such as?} 