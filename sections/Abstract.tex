Linear correlations between the spiking activity of pairs of neurons are among the most familiar and useful descriptive statistics of neural activity.
Multineuronal recordings provide far richer information than the equivalent number of neuron pairs considered in isolation.
For example, the full covariance matrix reveals the partial correlations between pairs of neurons or the common correlated activity involving the entire population.
Usual estimates of covariance matrix parameters require large samples for convergence.
Convergence can be improved by regularization, i.e.~by imposing some kind of structure.  
The optimal choice of the regularization approach is largely an empirical question as it depends profoundly on how well the imposed structure can still capture the dominant underlying physiogical interactions.
For example, presumed diffuse correlated input onto many neurons motivate low-rank parameterizations such as in factor analysis.  
Conversely, presumed strong pairwise linear associations between neurons motivate regularization by sparsifying the inverse covariance matrix.
Finally, `sparse+low-rank' inverse covariance parameterizations allow combining diffuse correlations and strong pairwise associations.  

To infer the optimal structure of neural correlations in a specitifc neural microcircuit, we compared the performance of several covariance estimators on the activity of 100--300 neurons in mouse visual cortex: sample covariance, Ledoit-Wolf covariance shrinkage, factor analysis, sparse inverse covariance, and sparse+low-rank inverse covariance.
The instantaneous firing rates, binned at 200 ms, were inferred from the somatic calcium signals acquired with fast 3D random-access two-photon microscopy.   
Each covariance estimator was optimized and evaluated by cross-validation.
Unsurprisingly, covariance shrinkage reliably outperformed the sample covariance estimate.
In turn, factor analysis-based estimates significantly outperformed covariance shrinkage.
Yet sparse inverse covariance with or without an additional low-rank component significantly outperformed both factor analysis and shrinkage  estimators.
The superior performance of the sparse inverse covariance estimator suggests the relative importance of detailed network interactions over common diffuse input in this circuit, although these results may vary in other circuits or in other brain states.