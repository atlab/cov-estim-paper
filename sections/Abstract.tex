Linear correlations between the spiking activity of pairs of neurons are among the most familiar and useful descriptive statistics of neural activity. Multineuronal recordings provide far richer information than the equivalent number of neuron pairs considered in isolation. For example, the full covariance matrix reveals the partial correlations between pairs of neurons, as well as correlated activity across the entire population. Covariance matrix estimates require large samples for convergence. Convergence  can be improved by regularization, \emph{i.e.}~by imposing some kind of structure.  The optimal regularization must be determined empirically since its performance depends on how the imposed structure captures the underlying interactions. For example, we can use low-rank parameterizations of the covariance matrix to account for correlated input onto many neurons. Conversely, if correlations are a result of a small number of  pairwise linear interactions between the observed neurons, we can sparsify the inverse covariance matrix. We can also use a `sparse+low rank’ inverse covariance representation to account for common fluctuations and pairwise interactions in the recorded population.  

To select the optimal structure of neural correlations in a local neural circuit, we compared the performance of several covariance estimators on the activity of 100--300 neurons in mouse visual cortex: sample covariance, Ledoit-Wolf covariance shrinkage, factor analysis, sparse inverse covariance, and sparse+low-rank inverse covariance. We inferred instantaneous firing rates in 200 ms bins from the somatic calcium signals acquired with fast 3D random-access two-photon microscopy.  Each covariance estimator was optimized and evaluated by cross-validation. As expected, covariance shrinkage reliably outperformed the sample covariance estimate. In turn, factor analysis-based estimates significantly outperformed covariance shrinkage. Yet sparse inverse covariance with or without an additional low-rank component significantly outperformed both factor analysis and shrinkage estimators. The superior performance of the sparse inverse covariance estimator suggests the relative importance of detailed network interactions over common diffuse input in the circuit we studied.
