Correlations \TODO{we assume readers understand that Pearson linear correlations are meant here} between the spiking activity of pairs of neurons are among the most familiar descriptive statistics of neural activity. Multineuronal recordings provide richer information than the equivalent number of neuron pairs. For example, the full covariance \TODO{we assume readers understand the relationship between covariance and correlation} matrix reveals correlated activity across the entire population as well as partial correlations between pairs.  Estimation of covariance matrices can be improved by regularization, \emph{i.e.}\;by imposing some kind of structure.  Optimal regularization must be determined empirically since its effect depends on how closely the imposed structure matches the underlying regularities. For example, we can use low-rank parameterizations of the covariance matrix to account for common fluctuations across the recorded population. Conversely, if correlations are strongly influenced by a small fraction of pairwise linear associations between the observed neurons, we can impose sparsity on the inverse covariance matrix. We can also use a `sparse+low rank' inverse covariance representation to account for both common fluctuations and pairwise interactions.  

To select the optimal structure of neural correlations in a local neural circuit, we compared the performance of several covariance estimators on the activity of 100--300 neurons in mouse visual cortex: sample covariance, covariance shrinkage, factor analysis, sparse inverse covariance, and sparse+low-rank inverse covariance. We inferred instantaneous firing rates in 200 ms bins from the somatic calcium signals acquired with fast 3D random-access two-photon microscopy.  Each covariance estimator was optimized and evaluated by cross-validation. As expected, covariance shrinkage reliably outperformed the sample covariance estimate. In turn, factor analysis-based estimates significantly outperformed covariance shrinkage. Yet sparse inverse covariance with or without an additional low-rank component significantly outperformed both factor analysis and shrinkage estimators. The superior performance of the sparse inverse covariance estimator suggests the relative importance of detailed network interactions over common diffuse input in the circuit we studied.
